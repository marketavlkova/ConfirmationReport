\chapter*{Conclusion}
\addcontentsline{toc}{chapter}{Conclusion}
\shorthandoff{-}

Prvotním cílem mé diplomové práce byla detekce antarktických izolátů spadajících do~rodu \tax{Pseudomonas}.
Pro tento účel byly využity dvě rodově specifické PCR, přičemž počáteční soubor obsahoval zástupce s biochemickým a fyziologickým profilem naznačujícím příslušnost k tomuto taxonu.
Výběr se tedy sestával z nesporulujících gramnegativních, glukózu nefermentujících tyček, tvořících převážně béžové, bělavé či~průhledné kolonie.
U~všech kmenů byla též detekována katalázová aktivita.
Produkce fluorescentních pigmentů taktéž poukazuje na to, že se může jednat o pseudomonádu.
Avšak absence tohoto znaku nebyla důvodem k vyloučení izolátu ze souboru presumptivních pseudomonád.
Následná PCR~selekce vyčlenila téměř polovinu kmenů z tohoto výběru mimo rod \tax{Pseudomonas}.
Nefluorescentní kmeny (KGB) příslušící dle této metody k pseudomonádám představují lehce nad 44\% zástupců.
Shluková a ordinační analýza fenotypu však poté ukázaly, že 28~izolátů zahrnutých mezi pseudomonády, se metabolicky značně liší od~kmenů typických pro~tento rod.
Genotypizace ale nespecifitu PCR použitých pro selekci nepotvrdila, jelikož těchto 28~zástupců se~nijak neodděluje od zbytku pseudomonád.
Naopak jsou rozptýleny v~rámci celého genotypizačního dendrogramu.
Dodatečná sekvenační analýza řadí vybrané zástupce taktéž do rodu \tax{Pseudomonas}.
K sekvenaci sice nebyl zvolen žádný z~takto fenotypově odlišných kmenů, ale na základě genotypizace se přikláním spíše k jejich příslušnosti mezi pseudomonády.

Co se týče fenotypizace, tak mimo hlavní soubor testů dle standardů České sbírky mikroorganismů, bylo provedeno dalších 12 dodatečných testů pro lepší charakterizaci.
Ukázalo~se, že převážná většina pseudomonád z Antarktidy okyseluje L-arabinózu, ribózu a~galaktózu a~naopak alkalizuje D-arabinózu a sacharózu.
Využití trehalózy je v~tomto případě velmi variabilní.
Celkem 25\% kmenů vykazuje elastázovou aktivitu, přičemž 1/4~z~nich byla izolována z tuleňů.
Růst na všech čtyřech testovaných mediích zvládá nadpoloviční většina zástupců.
Nejnižší procento růstu (77,5\%) je dle očekávání na~Cetrimidovém agaru a~nejvyšší (99,2\%) na~PCA.
Na mS1 agaru, jenž má sloužit k selekci kmenů \tax{Pseudomonas}~spp., neroste necelých 9\% našich izolátů.
Přitom pouze 2 kmeny (P3927 a~P4815) se~nenachází ve~fenotypově pro pseudomonády neobvyklém \textbf{\textcolor[RGB]{255, 0, 0}{Fenonu 1}}.
Pozoruhodné je~též~zjištění, že produkce fluorescentních pigmentů na King B mediu v~mnoha případech nekoreluje s~fluorescencí na mS1 agaru.
Pokusila jsem se též o druhovou identifikaci na~základě výsledků analýzy fenotypových znaků.
Nicméně následná genotypizace i~sekvenace konzervativních genů ukázaly, že v případě našeho souboru izolátů na~tento přístup nelze spoléhat.
Ačkoli shluková a ordinační analýza s celkem padesáti proměnnými taktéž zřejmě neodpovídá fylogenetickému postavení testovaných kmenů, poskytují nám alespoň informaci o~jejich metabolické diverzitě.
Zajímavé je zejména vyčlenění izolátů nacházejících~se ve~\textbf{\textcolor[RGB]{255, 0, 0}{Fenonu 1}}.

Jak jsem se již zmínila výše, genotypizace s využitím repetitivní PCR nastínila, že~klasickou biochemickou klasifikací nejsme schopni pseudomonády z~Antarktidy určit do~druhů.
Navíc se zdá, že ani celkové porovnání jejich metabolické aktivity neudává informaci o~jejich fylogenetických vztazích.
Zda však analýza pomocí rep-PCR odpovídá skutečné příbuznosti testovaných kmenů nelze zatím s jistotou říci.

Jelikož jsme stáli před dvěma typizačními metodami, které~si vzájemně v~podstatě neodpovídají, bylo sekvenování konzervativních genů vybraných izolátů stanoveno jako dodatečný cíl.
Náhodně jsme tedy vybrali 7 kmenů identifikovaných dle~fenotypu do~druhu \tax{P. fluorescens}, nacházejících se v rámci odlišných klastrů ve \hyperlink{genotyp.1}{Shlukové analýze genotypu}.
Pro~tuto analýzu byl ještě vybrán jeden neidentifikovaný zástupce.
Druhová identifikace však nebyla u šesti z celkem osmi izolátů vyřešena ani sekvenační analýzou.
V~dostupných databázích se nenachází druhy, které by s jistotou odpovídali těmto šesti kmenům.
Pro~jejich zařazení k~pseudomonádám v podobě nových druhů je nicméně třeba ještě další a~podrobnější charakterizace.

Výsledky této práce byly publikovány formou ústní prezentace v~rámci Students in~Polar and~Arctic Research Conference~2016 (viz~\hyperlink{SPARC.1}{Příloha}).
Příspěvek opět ve~formě ústního sdělení je přijat také na konferenci XXV. Tomáškovy dny mladých mikrobiologů konající se 2.~-~3.~června~2016.

\shorthandon{-}
\cleardoublepage

