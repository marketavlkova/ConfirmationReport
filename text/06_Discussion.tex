\chapter{Discussion}
%\addcontentsline{toc}{chapter}{Discussion}

\section{Specifita selekce pomocí PCR}
Z celkového počtu 439 presumptivních pseudomonád bylo pomocí PCR amplifikace genu~\tax{rpoD} a~variabilní části genu \tax{rrs} zahrnuto 240 kmenů do užšího výběru, který by~měl zahrnovat pouze zástupce rodu \tax{Pseudomonas}.
Avšak ne u všech izolátů sem~zahrnutých byly jednoznačně produkovány oba amplikony.
V případě 23~kmenů došlo k~pozitivní reakci pouze u~jednoho genu a~to i~v~případě opakovaného procesu.
U~22~zástupců byl detekován jen amplikon pro část genu \tax{rrs} a u jednoho byl přítomen pouze amplikon o~specifické délce genu~\tax{rpoD} s~negativní reakcí pro~\tax{rrs}~gen.
Tudíž tento soubor 240~kmenů teoreticky může zahrnovat i~zástupce mimo rod \tax{Pseudomonas}.

Během optimalizace obou PCR však byla testována specifita použitých primerů s~využitím 25~gramnegativních bakterií mimo~rod \tax{Pseudomonas}.
Níže uvádím Tabulku \ref{kontroly_PCR} s~výsledky reakcí.
PCR pro amplifikaci části genu \tax{rrs}, vytváří amplikon o délce odpovídající specifickému produktu i u kmene \tax{Azotobacter chroococcum} CCM 1912.
Bakterie rodu \tax{Azotobacter} jsou fylogeneticky pseudomonádám velmi blízké, avšak vyžadují pro svůj růst v~laboratorních podmínkách speciální media.
Žádné z těchto medií nebylo použito k~izolaci kmenů, tudíž je prakticky vyloučeno aby se zde tito zástupci vyskytovali.
Přítomnost nepříliš jasného amplikonu u typového kmene \tax{Raoultella terrigena} CCM 3568$^T$ také není velkým problémem.
Žádný z pracovních kmenů totiž není schopen fermentace glukózy, zatímco \tax{R. terrigena} je fakultativně anaerobní druh.
Přesto zjištění, že dané primery tvoří produkty o stejné délce jako specifické amplikony bylo překvapením.
V publikaci, ze~které jsem sekvence primerů čerpala, není o této nespecificitě zmínka. \cite{nair2014molecular}

Podobně i při použití primerů pro amplifikaci genu \tax{rpoD} jsme narazili na dvě falešně pozitivní reakce.
Šlo o 2 druhy aeromonád ze tří porovnávaných.
Tato nespecificita pro~nás ani tentokrát nečiní překážku v~použití primerů k~selekci pseudomonád, jelikož aeromonády jsou schopny fermentace glukózy, kdežto náš soubor již před tříděním obsahoval pouze nefermentující kmeny.
Původní publikace se opět o produkci amplikonů u~aeromonád s~délkou odpovídající specifickým produktům nezmiňuje. \cite{mulet2009rpod}
Avšak je zde poukázáno na možnou nespecificitu ve vztahu k rodu \tax{Alcanivorax} (z~řádu \tax{Oceanospirillales}).
Fenotypem se podobá pseudomonádám.
Má taktéž pouze respiratorní metabolismus, gramnegativní typ buněčné stěny, redukuje nitráty a~vykazuje katalázovou i~oxidázovou aktivitu.
Kolonie jsou bezbarvé. \cite{garrity2005bergey}
Jelikož v~České sbírce mikroorganismů není tento taxon k dispozici, nahradila jsem jej pro~tyto účely fylogeneticky příbuzným typovým kmenem \tax{Halomonas sulfidaeris} CCM 7108$^T$.
K~nespecifické reakci však u tohoto kmene nedošlo.

\begin{center}
\begin{longtable}[c]{|l|l|c|c|}
\caption{Referenční kmeny použité pro kontrolu specifity PCR s výsledky amplifikace} \label{kontroly_PCR} \\

\toprule \multicolumn{1}{|l|}{\textbf{Řád}} & \multicolumn{1}{l|}{\textbf{Referenční kmen}} & \multicolumn{1}{c|}{\textbf{\tax{rrs}}} & \multicolumn{1}{c|}{\textbf{\tax{rpoD}}}\\
\midrule
\endhead

\bottomrule
\endlastfoot

\tax{Rhizobiales} & \tax{Rhizobium radiobacter} CCM 2928 & - & -\\
\hline
\tax{Sphingomonadales} & \tax{Sphingomonas paucimobilis} CCM 3293 & - & -\\
\hline
\multirow{2}{*}{\tax{Burkholderiales}} & \tax{Alcaligenes faecalis} CCM 1052 & - & -\\
& \tax{Burkholderia cepacia} CCM 2656 & - & -\\
\hline
\multirow{3}{*}{\tax{Aeromonadales}} & \tax{Aeromonas dhakensis} CCM 7146 & - & +\\
& \tax{Aeromonas fluvialis} CCM 8437$^T$ & - & -\\
& \tax{Aeromonas hydrophila} CCM 2278 & - & +\\
\hline
\multirow{10}{*}{\tax{Enterobacteriales}} & \tax{Budvicia aquatica} CCM 3714 & - & -\\
& \tax{Citrobacter murliniae} CCM 4834$^T$ & - & -\\
& \tax{Enterobacter cloacae} CCM 1903 & - & -\\
& \tax{Erwinia amylovora} CCM 1114$^T$ & - & -\\
& \tax{Escherichia coli} CCM 5172$^T$ & - & -\\
& \tax{Hafnia alvei} CCM 4845 & - & -\\
& \tax{Klebsiella pneumoniae} CCM 7798$^T$ & - & -\\
& \tax{Rahnella aquatilis} CCM 4086 & - & -\\
& \tax{Raoultella terrigena} CCM 3568$^T$ & w & -\\
& \tax{Tatumella terrea} CCM 4324$^T$ & - & -\\
\hline
\tax{Oceanospirillales} & \tax{Halomonas sulfidaeris} CCM 7108$^T$ & - & -\\
\hline
\multirow{3}{*}{\tax{Pseudomonadales}} & \tax{Acinetobacter bohemicus} CCM 8462$^T$ & - & -\\
& \tax{Azotobacter chroococcum} CCM 1912 & + & -\\
& \tax{Psychrobacter faecalis} CCM 7339$^T$ & - & -\\
\hline
\multirow{2}{*}{\tax{Vibrionales}} & \tax{Vibrio alginolyticus} CCM 3578$^T$ & - & -\\
& \tax{Vibrio metschnikovii} CCM 7098 & - & -\\
\hline
\multirow{2}{*}{\tax{Xanthomonadales}} & \tax{Stenotrophomonas maltophilia} CCM 1640$^T$ & - & -\\
& \tax{Xanthomonas vesicatoria} CCM 2102 & - & -\\
\end{longtable}
\footnotesize
	\emph{Vysvětlivky:} + = pozitivní reakce; - = negativní reakce; w = přítomnost nejasného amplikonu\\*
	\end{center}

I přes zmíněné nespecificity jsme se rozhodli kombinaci těchto primerů použít pro~selekci psudomonád z počátečního souboru suspektních kmenů.
Ve fenotypizaci se nám následně vyčlenila skupina 28 izolátů (\textbf{\textcolor[RGB]{255, 0, 0}{Fenon 1}}), lišících se svými znaky od~hlavního shluku.
Mezi těmito kmeny by se mohly vyskytovat zástupci jiných rodů než \tax{Pseudomonas}.
Nasvědčuje tomu i fakt, že 90\%~zástupců nerostoucích na mS1 agaru, jenž má sloužit k~selekci pseudomonád z~prostředí \cite{gould1985new, tarnawski2003examination} se v~tomto fenonu nachází.
Ten obsahuje též~nadpoloviční většinu zástupců určených pouze jednou pozitivní uniplex~PCR mezi~pseudomonády.
Nicméně následná genotypizace tuto myšlenku nepotvrzuje, jelikož \textbf{\textcolor[RGB]{255, 0, 0}{Fenon 1}} roztříštila do mnoha shluků v celém dendrogramu.
S jistotou však vyloučit přítomnost jiných rodů než \tax{Pseudomonas} v~celém souboru 240~kmenů zatím nelze.

Co se týče falešně negativních výsledků a tedy chybného vyloučení izolátů z pseudomonád, opírám se především o testy citlivosti prováděné v rámci zmiňovaných studií. \cite{mulet2009rpod, nair2014molecular}
Chybovost také snižuje použití dvou rodově specifických reakcí místo jedné.
Tudíž u kmenů s pozitivní pouze jednou PCR se můžeme cíleně zaměřit na potvrzení či vyvrácení jejich příslušnosti k pseudomonádám.



\cleardoublepage
