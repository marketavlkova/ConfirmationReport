\chapter{Materials and Methods}
%\setcounter{page}{1}
%\pagenumbering{arabic}
%\addcontentsline{toc}{chapter}{Metody}


%%%%%%%%%%%%%%%%%%%%%%%%%%%%%%%%%%%%
%%%%%%%%% GENERUJ TEXT %%%%%%%%%%%%%

\shorthandoff{-} 

\section{Strains and growth conditions}
All strains used in this work belong to the genus \tax{Escherichia coli} and are derived from a classical laboratory strain K12 (MG1655) and an environmental strain SC1\textunderscore D9 isolated from a bank of St. Louis River in Minnesota within a great sampling study \cite{ishii2006presence}.
In future work additional environmental isolates are likely to be used.

\begin{center}
	\begin{longtable}[c]{|l|c|c|c|c|}
\caption{Strains used in the study so far} \label{strains} \\

\toprule \multicolumn{1}{|l|}{\textbf{Strain ID}} & \multicolumn{1}{c|}{\textbf{Relevant genotype}} & \multicolumn{1}{c|}{\textbf{Promoter}} & \multicolumn{1}{c|}{\textbf{Ancestor}} & \multicolumn{1}{c|}{\textbf{Reference}} \\
\midrule
\endhead

\bottomrule
\endlastfoot

MG1655 & \tax{E. coli} K12 wild-type & - & - & \cite{blattner1997complete} \\
\hline
SC1\textunderscore D9 & \textcolor{red}{?} & - & - & \textcolor{red}{?} \\
%%% should I use the same reference as above? (ishii2006presence)
\hline
pUA66-K12 & pUA66 & - & MG1655 & \cite{zaslaver2006comprehensive} \\
\hline
pU139-K12 & pU139 & - & MG1655 & \cite{zaslaver2006comprehensive} \\
\hline
pU139-D9 & pU139 & - & SC1\textunderscore D9 & this study \\
\hline
ASC662 & \tax{lacZ-GFP} & - & MG1655 & \cite{kiviet2014stochasticity} \\
\hline
pZ1\textunderscore K12-K12 & \tax{placZ::GFP} & MG1655 & MG1655 & \cite{zaslaver2006comprehensive} \\
\hline
pZ2\textunderscore K12-K12 & \tax{placZ::GFP} & MG1655 & MG1655 & this study \\
\hline
pZ2\textunderscore K12-D9 & \tax{placZ::GFP} & MG1655 & SC1\textunderscore D9 & \textcolor{red}{?} \\
\hline
pZ\textunderscore D9-D9 & \tax{placZ::GFP} & SC1\textunderscore D9 & SC1\textunderscore D9 & \textcolor{red}{?} \\
\hline
pZ\textunderscore D9-K12 & \tax{placZ::GFP} & SC1\textunderscore D9 & MG1655 & \textcolor{red}{?} \\
\hline
pZm1\textunderscore D9-K12 & \tax{placZm172::GFP} & SC1\textunderscore D9 & MG1655 & \textcolor{red}{?} \\
\hline
pZm2\textunderscore D9-K12 & \tax{placZm279::GFP} & SC1\textunderscore D9 & MG1655 & \textcolor{red}{?} \\
\hline
pA\textunderscore K12-K12 & \tax{precA::GFP} & MG1655 & MG1655 & \cite{zaslaver2006comprehensive} \\
\hline
pA\textunderscore K12-D9 & \tax{precA::GFP} & MG1655 & SC1\textunderscore D9 & this study \\
\hline
Top10 & \text{*} & - & MG1655 & \textcolor{red}{?} \\
\hline
pLW001-Top10 & pLW001 & - & Top10 & \textcolor{red}{?} \\
	\end{longtable}
\footnotesize
	\emph{\text{*}} F– mcrA $\Delta$(mrr-hsdRMS-mcrBC) $\Phi$80lacZ$\Delta$M15 $\Delta$lacX74 recA1 araD139 $\Delta$(ara leu)7697 galU galK rpsL(Str$^{R}$) endA1 nupG\\*
%%% I took it from Quartzy and found out it differs a bit from E. coli genotypes description in https://openwetware.org/wiki/E._coli_genotypes#TOP10_.28Invitrogen.29
%%% is it correct???
\end{center}

For flow cytometry and microscopy purposes M9 minimal medium [Sigma-Aldrich] supplemented with 0.4\% of a single carbon source was used.
As those D(+)-glucose [LabServ], lactose [Ajax Finechem], D(+)-galactose [Acros Organics], L-arabinose [Alfa Aesar] and D(-)-ribose [Acros Organics] were chosen.
In other cases in LB medium [Invitrogen] (liquid or 1.5\% agar [BD]) was used.
Kanamycin [PanReac AppliChem] to concentration 50$\mu$g/ml was also added to the media if the bacterium contained pU139 or pUA66 plasmid (with or without promoter upstream of GFP).
All strains were grown at 37$^{\circ}$C, except for pLW001-Top10 which was grown at 30$^{\circ}$C (and presence of 100$\mu$g/ml Ampicillin [PanReac AppliChem] as well).

\section{Transformation}
Desired plasmids for transformation were isolated from donor strains grown in liquid LB supplemented with appropriated antibiotic and using StrataPrep Plasmid Miniprep Kit according to manufacture's instructions.
The only exceptions were using sterile deionized water instead of elution buffer and growing pLW001-Top10 strain for 24h at 30$^{\circ}$C prior to the very plasmid isolation.
If the concentration of acquired plasmid DNA was too low, it was concentrated using centrifugal vacuum concentrator [Eppendorf\textsuperscript{\textregistered} Basic Model 5301] at room temperature.

\subsection{Preparing electrocompetent cells}
A flask of 1l capacity containing 100ml of liquid LB was inoculated by 1ml of overnight culture (or appropriate volumes) and incubated at 37$^{\circ}$C with shaking.
After reaching OD$_{600}$ between 0.5 -- 0.9 the culture was fast chilled by swirling in ice slurry for 5 -- 10min and then kept in the slurry for 1h.
Next the culture was transferred into 2 pre-chilled 50ml falcon tubes and centrifuged  at 4$^{\circ}$C for 10min at 4200G.
The supernatant was discarded and pellet was washed twice in about 25ml of 10\% glycerol [\textcolor{red}{?}] with centrifugation at 4$^{\circ}$C for 10min at 4500G.
The final supernatant (about 0.5ml when starting with 100ml culture) was pooled into one falcon tube and aliquoted into pre-chilled (in -80$^{\circ}$C freezer) 1.7ml tubes.

\subsection{Electroporation}
A tube of electrocompetent cells was thawed on ice and 1$\mu$l of desired plasmid was added to it.
The cells were then transferred into 0.1cm electroporation cuvette [BioRad] and electroporated at 1850V [Eppendorf Eporator\textsuperscript{\textregistered}].
500ml of SOC media (\textcolor{red}{content}) was added immediately to the cuvette and the whole content then transferred into pre-warmed 5ml tube.
After incubation for 1h at 37$^{\circ}$C (or 2h at 30$^{\circ}$C in case of pLW001 plasmid background) with shaking, 10-fold dilutions up to 10$^{-3}$ were prepared and spread by sterile beads on LB agar plates containing appropriate antibiotic as well as non-diluted culture.
Plates were then incubated overnight at 37$^{\circ}$C (or two overnights at 30$^{\circ}$C) and single colonies were checked for desired plasmid presence by PCR.

\subsection{TSS transformation}
A flask of 100ml capacity containing 10ml of liquid LB was inoculated by 0.1ml of overnight MG1655 culture (for other strains electroporation was used) and incubated at 37$^{\circ}$C with shaking.
When OD$_{600}$ rose to values between 0.5 -- 0.9 the culture was fast chilled by swirling in ice slurry for 15min.
Next 0.75ml of the chilled culture was transferred into 5ml tube containing the same amount of ice cold 2x TSS (\textcolor{red}{content}) and was incubated on ice slurry for 30min.
2$\mu$l of desired plasmid was then added to the cells with subsequent incubation on ice slurry for another 1h.
After additional incubation for 1h at 37$^{\circ}$C with shaking, cells were centrifuged at maximum speed (i.e. 14000rpm = 20817G) for 5min at room temperature, supernatant was discarded and pellet re-suspended in 0.2ml of LB.
From this 10-fold dilutions were prepared up to 10$^{-3}$ and these were spread by sterile beads on LB agar plates containing appropriate antibiotic as well as non-diluted culture.
Plates were then incubated overnight at 37$^{\circ}$C and single colonies were checked for desired plasmid presence by PCR.

\section{Activity assays of \tax{lacZ} promoter}


\shorthandon{-} 
%%%%%%%%%%%%%%%%%%%%%%%%%%%%%%%%%%%%
