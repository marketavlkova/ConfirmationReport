\chapter{Materials and Methods}
%\setcounter{page}{1}
%\pagenumbering{arabic}
%\addcontentsline{toc}{chapter}{Metody}


%%%%%%%%%%%%%%%%%%%%%%%%%%%%%%%%%%%%
%%%%%%%%% GENERUJ TEXT %%%%%%%%%%%%%

\shorthandoff{-} 

\section{Strains and growth conditions}
All strains used in this work belong to the genus \tax{Escherichia coli} and are derived from a classical laboratory strain K12 (MG1655) and an environmental strain SC1\textunderscore D9 isolated from a bank of St. Louis River in Minnesota within a great sampling study \cite{ishii2006presence}.
In future work additional environmental isolates are likely to be used.

\begin{center}
	\begin{longtable}[c]{|l|c|c|c|c|}
\caption{Strains used in the study so far} \label{strains} \\

\toprule \multicolumn{1}{|l|}{\textbf{Strain ID}} & \multicolumn{1}{c|}{\textbf{Relevant genotype}} & \multicolumn{1}{c|}{\textbf{Promoter}} & \multicolumn{1}{c|}{\textbf{Ancestor}} & \multicolumn{1}{c|}{\textbf{Reference}} \\
\midrule
\endhead

\bottomrule
\endlastfoot

MG1655 & \tax{E. coli} K12 wild-type & - & - & \cite{blattner1997complete} \\
\hline
SC1\textunderscore D9 & ? & - & - & ? \\
%%% should I use the same reference as above? (ishii2006presence)
\hline
pUA66-K12 & pUA66 & - & MG1655 & \cite{zaslaver2006comprehensive} \\
\hline
pU139-K12 & pU139 & - & MG1655 & \cite{zaslaver2006comprehensive} \\
\hline
pU139-D9 & pU139 & - & SC1\textunderscore D9 & this study \\
\hline
ASC662 & \tax{lacZ-GFP} & - & MG1655 & \cite{kiviet2014stochasticity} \\
\hline
pZ1\textunderscore K12-K12 & \tax{placZ::GFP} & MG1655 & MG1655 & \cite{zaslaver2006comprehensive} \\
\hline
pZ2\textunderscore K12-K12 & \tax{placZ::GFP} & MG1655 & MG1655 & this study \\
\hline
pZ2\textunderscore K12-D9 & \tax{placZ::GFP} & MG1655 & SC1\textunderscore D9 & ? \\
\hline
pZ\textunderscore D9-D9 & \tax{placZ::GFP} & SC1\textunderscore D9 & SC1\textunderscore D9 & ? \\
\hline
pZ\textunderscore D9-K12 & \tax{placZ::GFP} & SC1\textunderscore D9 & MG1655 & ? \\
\hline
pZm1\textunderscore D9-K12 & \tax{placZm172::GFP} & SC1\textunderscore D9 & MG1655 & ? \\
\hline
pZm2\textunderscore D9-K12 & \tax{placZm279::GFP} & SC1\textunderscore D9 & MG1655 & ? \\
\hline
pA\textunderscore K12-K12 & \tax{precA::GFP} & MG1655 & MG1655 & \cite{zaslaver2006comprehensive} \\
\hline
pA\textunderscore K12-D9 & \tax{precA::GFP} & MG1655 & SC1\textunderscore D9 & this study \\
\hline
Top10 & \text{*} & - & MG1655 & ? \\
\hline
pLW001-Top10 & pLW001 & - & Top10 & ? \\
	\end{longtable}
\footnotesize
	\emph{\text{*}} F– mcrA $\Delta$(mrr-hsdRMS-mcrBC) $\Phi$80lacZ$\Delta$M15 $\Delta$lacX74 recA1 araD139 $\Delta$(ara leu)7697 galU galK rpsL(Str$^{R}$) endA1 nupG\\*
%%% I took it from Quartzy and found out it differs a bit from E. coli genotypes description in https://openwetware.org/wiki/E._coli_genotypes#TOP10_.28Invitrogen.29
%%% is it correct???
\end{center}

For flow cytometry and microscopy purposes M9 minimal medium [] supplemented with 0.4\% of a single carbon source was used.
As those D(+)-glucose [], lactose [], D(+)-galactose [], L-arabinose [] and D(-)-ribose [] were chosen.
In other cases in LB medium [] (liquid or 1.5\% agar []) was used.
Kanamycin [] to concentration 50$\mu$g/ml was also added to the media if the bacterium contained pU139 or pUA66 plasmid (with or without promoter upstream of GFP).
All strains were grown at 37$^{\circ}$C, except for pLW001-Top10 which was grown at 30$^{\circ}$C (and presence of 100$\mu$g/ml Ampicillin [] as well).

\section{Transformation}
\subsection{Preparing electrocompetent cells}
A flask of 1l capacity containing 100ml of liquid LB was inoculated by 1ml of overnight culture (or appropriate volumes) and incubated at 37$^{\circ}$C with shaking.
After reaching OD$_{600}$ between 0.5 -- 0.9 the culture was fast chilled by swirling in ice slurry for 5 -- 10min and then kept in the slurry for 1h.
Next the culture was transferred into 2 pre-chilled 50ml falcon tubes and centrifuged  at 4$^{\circ}$C for 10min at 4200G.
The supernatant was discarded and pellet was washed twice in about 25ml of 10\% glycerol [] with centrifugation at 4$^{\circ}$C for 10min at 4500G.
The final supernatant (about 0.5ml when starting with 100ml culture) was pooled into one falcon tube and aliquoted into pre-chilled (in -80$^{\circ}$C freezer) 1.7ml tubes.

\subsection{Electroporation}
A tube of electrocompetent cells was thawed on ice and 1$\mu$l of desired plasmid was added to it.
The cells were then transferred into 0.1cm electroporation cuvette [BioRad] and electroporated at 1850V [Eppendorf Eporator\textsuperscript{\textregistered}].
500ml of SOC media () was added immediately to the cuvette and the whole content then transferred into pre-warmed 5ml tube.
After incubation for 1h at 37$^{\circ}$C (or 2h at 30$^{\circ}$C in case of pLW001 plasmid background) with shaking, 10-fold dilutions up to 10$^{-3}$ were prepared and spread by beads on LB agar plates containing appropriate antibiotic as well as non-diluted culture.
Plates were then incubated overnight at 37$^{\circ}$C (or two overnights at 30$^{\circ}$C) and single colonies were checked for desired plasmid presence by PCR.

\section{Fenotypové metody}
\subsection{Morfologie buněk a kolonií}
Popis morfologie buněk, Gramovo barvení, zapsání tvaru a barvy kolonií proběhlo dle~běžných standardů České sbírky mikroorganismů.

\subsection{Biochemické testy}
Základních 34 testů u všech gramnegativních nefermentujících kmenů z Antarktidy bylo provedeno dle standardních postupů České sbírky mikroorganismů.
Na základě těchto výsledků se vytypovali presumptivní zástupci pseudomonád, kteří byli dále konfirmováni pomocí uniplex PCR.

Dodatečné testy k detekci exprese enzymů a využívání dalších látek jsem prováděla u~240 kmenů, jenž byly vyhodnoceny dle uniplex PCR jako \tax{Pseudomonas} sp.
Pro stanovení přítomnosti či absence daného znaku jsem použila klasické zkumavkové a miskové testy.
Kultivace probíhala v 15$^{\circ}$C či 30$^{\circ}$C (dle optimálního růstu jednotlivých kmenů).

\subsubsection{Okyselování a alkalizace cukrů}
Testovala jsem schopnost jednotlivých kmenů využívat L-arabinózu, D-arabinózu, ribózu, galaktózu, trehalózu a sacharózu jako zdroj uhlíku a energie (acidifikace media).

Polotuhé medium jsem připravila přidáním 10~ml zásobního 10\% cukerného roztoku předem vysterilizovaného při 118$^{\circ}$C po 20 minut do 90~ml sterilního OF bazálního media [Merck] (11~g/l; sterilizace: 121$^{\circ}$C, 15~min.) ochlazeného na cca 60$^{\circ}$C.
Po důkladném promíchání jsem roztok ještě za tepla rozplnila do sterilních zkumavek.

Bakteriální kmeny jsem očkovala vpichem.
Výsledky se odečítaly po 1 až 8 dnech kultivace.
Tvorba kyselin se projevila změnou barvy původního neutrálního media ze~zelené na~žlutou.
V případě alkalizace se medium zbarvilo do modra.
Jako pH indikátor zde sloužila bromtymolová modř.

\subsubsection{Hydrolýza elastinu}
Jde o test na detekci produkce extracelulárního enzymu elastázy, jenž může být znakem virulence.

Použité medium se sestávalo z 3~g elastinu s kongočervení a 20~g agaru [HiMedia] na 1~litr BHI (brain-heart infusion) [Bio-Rad].
Po 15 minutové sterilizaci při 121$^{\circ}$C jsem medium rozlila do Petriho misek.

Na misku bylo pomocí bakteriologické kličky očkováno po osmi kmenech.
Odečet výsledků probíhal po 1 až 14 dnech kultivace.
Hydrolýza elastinu se projevila projasněním media v místě růstu bakterie příp. i v jejím okolí.
Zaznamenávala jsem též černání media v~místě nárůstu kmenů.

\shorthandon{-} 
%%%%%%%%%%%%%%%%%%%%%%%%%%%%%%%%%%%%
