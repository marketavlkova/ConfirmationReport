\chapter{Materials and Methods}
%\setcounter{page}{1}
%\pagenumbering{arabic}
%\addcontentsline{toc}{chapter}{Metody}


%%%%%%%%%%%%%%%%%%%%%%%%%%%%%%%%%%%%
%%%%%%%%% GENERUJ TEXT %%%%%%%%%%%%%

\shorthandoff{-} 

\section{Původ, sběr a zpracování vzorků}
K práci byly použity izoláty uložené v České sbírce mikroorganismů jako pracovní kmeny.
Jednalo se o soubor vzorků pocházejících z Antarktidy, jenž byl vyizolován v~letech 2011 až 2014.

U abiotických vzorků byla odebírána především zemina a voda.
Z vody se naočkovávalo přímo 100 $\mu$l vzorku na misku s R2A agarem [Oxoid] bez ředění.
Vzorky půdy a horniny byly suspendovány ve 3 ml fyziologického roztoku.
Poté bylo provedeno desetinné ředění až do 10$^{-7}$ a následoval opět výsev jako u vzorků vody.
Narostlé kmeny s rozmanitou makroskopickou morfologií byly
odpichovány z misek s R2A agarem po kultivaci v 15$^\circ$C do křížového roztěru pro posouzení čistoty kultury.

Biotické vzorky pocházely z tuleňů.
Jednalo se především o výtěry z tlamy a rekta do~transportního media Amies [Dispolab].
Kmeny byly izolovány na BHI agarech [Bio-Rad] po kultivaci v 30$^\circ$C.

Všechny tyto kroky (sběr vzorků, výsev a izolace čistých kultur) probíhaly dle zavedených a standardních postupů České sbírky mikroorganismů.


\section{Uchovávání izolátů}
Čisté izolované kmeny i použité referenční kmeny (CCM) byly uchovávány v hlubokomrazících boxech při -70$^{\circ}$C na keramických nosičích (korálcích) v polypropylenových zkumavkách.

Testované antarktické kmeny jsem z hlubokomrazících boxů oživovala přenesením jednoho korálku vyžíhanou kličkou do 0,5 ml bujónu ve zkumavce se šikmým R2A agarem [Oxoid].
Bujón se následně nechal stéci po celé ploše agaru a zkumavky se daly inkubovat na~2~až~7~dnů do~15$^{\circ}$C.

\section{Fenotypové metody}
\subsection{Morfologie buněk a kolonií}
Popis morfologie buněk, Gramovo barvení, zapsání tvaru a barvy kolonií proběhlo dle~běžných standardů České sbírky mikroorganismů.

\subsection{Biochemické testy}
Základních 34 testů u všech gramnegativních nefermentujících kmenů z Antarktidy bylo provedeno dle standardních postupů České sbírky mikroorganismů.
Na základě těchto výsledků se vytypovali presumptivní zástupci pseudomonád, kteří byli dále konfirmováni pomocí uniplex PCR.

Dodatečné testy k detekci exprese enzymů a využívání dalších látek jsem prováděla u~240 kmenů, jenž byly vyhodnoceny dle uniplex PCR jako \tax{Pseudomonas} sp.
Pro stanovení přítomnosti či absence daného znaku jsem použila klasické zkumavkové a miskové testy.
Kultivace probíhala v 15$^{\circ}$C či 30$^{\circ}$C (dle optimálního růstu jednotlivých kmenů).

\subsubsection{Okyselování a alkalizace cukrů}
Testovala jsem schopnost jednotlivých kmenů využívat L-arabinózu, D-arabinózu, ribózu, galaktózu, trehalózu a sacharózu jako zdroj uhlíku a energie (acidifikace media).

Polotuhé medium jsem připravila přidáním 10~ml zásobního 10\% cukerného roztoku předem vysterilizovaného při 118$^{\circ}$C po 20 minut do 90~ml sterilního OF bazálního media [Merck] (11~g/l; sterilizace: 121$^{\circ}$C, 15~min.) ochlazeného na cca 60$^{\circ}$C.
Po důkladném promíchání jsem roztok ještě za tepla rozplnila do sterilních zkumavek.

Bakteriální kmeny jsem očkovala vpichem.
Výsledky se odečítaly po 1 až 8 dnech kultivace.
Tvorba kyselin se projevila změnou barvy původního neutrálního media ze~zelené na~žlutou.
V případě alkalizace se medium zbarvilo do modra.
Jako pH indikátor zde sloužila bromtymolová modř.

\subsubsection{Hydrolýza elastinu}
Jde o test na detekci produkce extracelulárního enzymu elastázy, jenž může být znakem virulence.

Použité medium se sestávalo z 3~g elastinu s kongočervení a 20~g agaru [HiMedia] na 1~litr BHI (brain-heart infusion) [Bio-Rad].
Po 15 minutové sterilizaci při 121$^{\circ}$C jsem medium rozlila do Petriho misek.

Na misku bylo pomocí bakteriologické kličky očkováno po osmi kmenech.
Odečet výsledků probíhal po 1 až 14 dnech kultivace.
Hydrolýza elastinu se projevila projasněním media v místě růstu bakterie příp. i v jejím okolí.
Zaznamenávala jsem též černání media v~místě nárůstu kmenů.

\shorthandon{-} 
%%%%%%%%%%%%%%%%%%%%%%%%%%%%%%%%%%%%
