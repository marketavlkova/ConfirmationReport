\renewcommand{\chaptername}{Introduction}
\chapter*{Introduction}
\setcounter{page}{1}
\pagenumbering{arabic}
\addcontentsline{toc}{chapter}{Introduction}


\shorthandoff{-} 

\section{Taxonomie rodu \tax{Pseudomonas}}
%\addcontentsline{toc}{section}{Taxonomie rodu \tax{Pseudomonas}}
               Doména:	\hspace{0,5cm} \tax{Bacteria}\\%
\hspace*{1,5cm} Kmen:	  \hspace{0,5cm} \tax{Proteobacteria}\\%
\hspace*{2,5cm} Třída:   \hspace{0,5cm} \tax{Gammaproteobacteria}\\%
\hspace*{3,5cm} Řád:     \hspace{0,5cm} \tax{Pseudomonadales}\\%
\hspace*{4,5cm} Čeleď:   \hspace{0,5cm} \tax{Pseudomonadaceae}\\%

Během posledních let došlo k výrazné změně taxonomie tohoto rodu, současně se~změnou bakteriální taxonomie jako takové.
Na základě nového přístupu členění bakterií do~rodů dle příbuznosti genu pro 16S rRNA bylo mnoho zástupců řazených pod \tax{Pseudomonas}~spp. reklasifikováno.
I přes to, že druhů z původního souboru příliš nezůstalo, tento rod patří s~více než dvěmi sty taxony k druhově nejbohatším \cite{www.bacterio.net}.
Zásluhu na tom má stále rostoucí počet nově popsaných zástupců. \cite{cornelis2008pseudomonas}

\subsection{Historie a současnost}
Objev a definice prvních pseudomonád se datuje do konce 19. století.
Následovalo velké množství nově popsaných druhů izolovaných z různých zdrojů.
V polovině 20.~století bylo k tomuto rodu řazeno odhadem přes 800 druhů. \cite{palleroni2010pseudomonas}
Páté vydání Bergeyho manuálu z roku 1957 již zahrnovalo tyto zástupce a jejich fenotypové vlastnosti.
Bohužel obsažené informace byly čerpány především z amerických studií a postrádaly několik důležitých zjištění, ke kterým přišli evropští badatelé.
Tak se například americká bakteriologická komunita dozvěděla o bohaté metabolické aktivitě pseudomonád až v 60.~letech, ačkoliv první práce popisující tyto charakteristiky je z roku 1926. \cite{palleroni2010pseudomonas}

Morfologie pseudomonád se příliš neliší od jiných i velmi fylogeneticky vzdálených rodů.
Tudíž vlastnosti jako tvar, velikost, uspořádání buněk, přítomnost inkluzí a pohyblivost byly doplňovány množstvím fyziologických testů - produkcí pigmentů, růstovými, nutričními a teplotními podmínkami, akumulací biopolymerů či testy produkce intra i extracelulárních enzymů.
Taxonomicky užitečná se ukázala schopnost produkce fluorescentních sideroforů.
Popis druhů byl posléze doplněn daty ohledně procentuálního zastoupení bází. \cite{cornelis2008pseudomonas, palleroni2010pseudomonas}

V 70. letech byla publikována práce, která velmi zredukovala dosavadní počet pseudomonád a následně ovlivnila i celou taxonomii bakterií. \cite{palleroni1973nucleic, frey1997molecular, clarridge2004impact, larsen2014benchmarking}
Bylo zjištěno, že zástupci zahrnovaní do rodu \tax{Pseudomonas} (nyní \tax{Pseudomonas sensu lato}), se dělí do pěti ostře ohraničených skupin dle podobnosti rRNA.
Jako \tax{Pseudomonas sensu stricto} pak byla vyčleněna rRNA skupina I, která obsahovala typový druh \tax{P. aeruginosa}, všechny fluorescentní a některé nefluorescentní druhy.
Navazující studie na jiných bakteriích prokázaly, že konzervovanost rRNA je velmi užitečná v rodové diferenciaci a taxonomie do té doby postavená především na morfologických a fyziologických znacích byla přehodnocena. \cite{ fox1977comparative, woese1980phylogenetic, ludwig1981phylogenetic}
Dnes nám již pro popis nových druhů nestačí pouze fenotypová charakteristika, ale je třeba přistoupit k tzv. polyfázové taxonomii. \cite{tindall2010notes, thompson2015microbial}
Fenotypovou charakteristiku zpravidla doplňují molekulární analýzy a chemotaxonomie, dokonce se jim povětšinou věnuje více pozornosti než klasické fenotypizaci.
Avšak ani bez jednoho přístupu se v dnešní komplexní taxonomii neobejdeme. \cite{palleroni1973nucleic, palleroni2010pseudomonas}

\section{Charakteristika rodu \tax{Pseudomonas}}
Typovým rodem čeledi \tax{Pseudomonadaceae} je \tax{Pseudomonas} Migula 1894.
Jedná se o~rovné či lehce zakřivené gramnegativní tyčinky, které netvoří klidová stádia a nehromadí granula poly-$\beta$-hydroxybutyrátu.
Zástupci tohoto rodu jsou povětšinou opatřeni jedním či~více polárními bičíky.
Mají striktně respiratorní metabolismus s molekulárním kyslíkem jako konečným akceptorem elektronů.
Nitrát však může v některých případech sloužit jako alternativní akceptor elektronů a umožňovat tak růst za mírně anaerobních podmínek.
Tito aerobní chemoorganotrofové obecně nevyžadují růstové faktory a charakteristická je pro~ně produkce katalázy, přičemž oxidázová aktivita nemusí být přítomna.
Typovým druhem tohoto rodu je \tax{Pseudomonas aeruginosa}. \cite{garrity2005bergey}

\subsection{Bičíky a fimbrie}
%\addcontentsline{toc}{subsection}{Bičíky a fimbrie}
Ačkoliv je pro pseudomonády typické polární uložení bičíků, v některých případech se mohou vyskytovat subpolárně či laterálně (\tax{P. stutzeri}, \tax{P. mendocina}).
Zato počet bičíků (resp. přítomnost právě jednoho či více než jednoho) je důležitým taxonomickým znakem.
U polárního monotricha \tax{P. aeruginosa} byly objeveny na lokusu fliC dva geny kódující flagelin, přičemž jeden z nich je variabilní. \cite{spangenberg1996genetic}

Přítomnost polárních fimbrií byla detekována u \tax{P. aeruginosa} a \tax{P. alcaligenes}.
Druhy \tax{P. fluorescens}, \tax{P. chlororaphis} a \tax{P. putida} naopak fimbrie (pili) postrádají.
Tyto struktury se podílejí např. na adhezi či trhavém pohybu buněk po pevném povrchu, ale mohou sloužit i jako receptory pro bakteriofágy. \cite{garrity2005bergey}



\cleardoublepage

