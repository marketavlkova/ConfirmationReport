\renewcommand{\chaptername}{Introduction}
\chapter*{Introduction}
\setcounter{page}{1}
\pagenumbering{arabic}
\addcontentsline{toc}{chapter}{Introduction}


\shorthandoff{-} 

\section{Taxonomy and characteristics of \tax{Escherichia coli}}
\hspace*{0,5cm} Domain: \hspace{0,5cm} \tax{Bacteria}\\%
\hspace*{1,5cm} Phylum: \hspace{0,5cm} \tax{Proteobacteria}\\%
\hspace*{2,5cm} Class: \hspace{0,5cm} \tax{Gammaproteobacteria}\\%
\hspace*{3,5cm} Order: \hspace{0,5cm} \tax{Enterobacteriales}\\%
\hspace*{4,5cm} Family: \hspace{0,5cm} \tax{Enterobacteriaceae}\\%

\tax{E. coli} is Gram-negative bacterium with facultative anaerobic metabolism.
Since its first isolation from infant faeces by Theodor Escherich in 1885 \cite{friedmann2006escherich}, this organism was found in gastrointestinal tract of other warm-blooded animals and reptiles \cite{gopee2000longitudinal} as well as in environmental samples such as soil or water.
\tax{E. coli} is known to be a part of normal intestine microbiota.
On the other hand extraintestinal infections caused by this bacterium are common and some strains (e.g. STEC) even possess virulence factors leading to heavy intestinal infections when expressed \cite{allocati2013escherichia}.


Although the presence of \tax{E. coli} in water is still widely used as an indicator of fecal contamination recent studies show that some \tax{E. coli} isolates are able to  reproduce in soil \cite{byappanahalli2004indigenous, somorin2016general}.
Moreover strains inhabiting soil for a long time were found to be distinctive \cite{walk2009cryptic, walk2015cryptic}.
This raises the questions how these bacteria survive outside their hosts and how do they evolve under such conditions.

\section{Transcription and memory}
A bacterial genome consists of hundreds and thousands of genes, but not all of them are active at all the time.
Gene expression changes during a cell cycle, when a nutrient source is altered or if surrounding conditions begin to be unfavourable.
It is vital for a living cell to sense what is happening in the environment it is present at and react accordingly by transcription initiation or silencing of appropriate genes.
Some studies has shown that cells with the same genetic background might react differently or at a different rate to the same stimulus based on their and their ancestors recent experience \cite{mathis2017asymmetric, ronin2017long}.
Such ability is advantageous especially if the change in the conditions is predictable and repeats periodically.

\subsection{Epigenetic mechanisms in bacteria}
Epigenetics is now understood as a heritable change in gene expression without simultaneous changes in DNA primary sequence.
This reversible altered expression is usually maintained by a positive feedback loop.

\subsubsection{DNA methylation}


\cleardoublepage

