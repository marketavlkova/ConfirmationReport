\chapter*{Objectives}
\addcontentsline{toc}{chapter}{Objectives}

\shorthandoff{-} 

I will use a collection of environmental \tax{E. coli} strains isolated near Lake Superior, MN, USA \cite{ishii2006presence}.
These strains are distributed across the whole phylogenetic tree of the species.
They will be used in order to elucidate how natural selection acts on transcriptional responses and epigenetic memory of several promoters by investigating both genotypic and phenotypic variation among them.

\begin{enumerate}[font=\bfseries]

    \item \textbf{Quantify genotypic and phenotypic variation in promoters among environmental \tax{E. coli} isolates}
    
    In order to quantify variation in transcriptional responses, we will focus initially on studying natural variation in the response of the \tax{lac} operon in the presence of different carbon sources, both separately and in combination.
    To begin, we will study these responses using a plasmid-based system with transcriptional fusion.
    Previously, initial data comparing environmental strain SC1\textunderscore D9 and a lab strain MG1655 were collected which suggested that differences in the \tax{lacZ} promoter sequence and further downstream in \tax{lacZ} gene itself result in differential gene expression.
    Moreover, not only the sequence but also the genetic background of the strain was observed to affect the level of expression (i.e., both cis- and trans-effects are present).
    I will follow up on and expand these findings with \tax{lacZ} promoter and include four other promoters and ten environmental \tax{E. coli} strains as well.
    The specific steps of this objective are described further below.

    \begin{enumerate}[font=\bfseries]
    
        \item \textbf{Select promoter sequences among environmental \tax{E. coli} isolates to study}
        
        To begin we have selected the well-studied \tax{lacZ} and \tax{recA} promoters, because each creates a different type of cell response - carbon source catabolism and stress response, respectively.
        We will select three other transcriptional circuits to study as well.
        Promoters with known function and regulation mechanism will be picked based on an annotated MG1655 genome.
        Then we will search for corresponding sequences in genomes of natural isolates and identify the sequence differences in them.
        As we want to cover a broad variation scale on the genetic level, we will select promoter sequences which are both highly and slightly diverse among natural isolates.
        Those which are mainly bound by RNA polymerase with a conventional $\sigma^{70}$ factor will be preferred.

        \item \textbf{Quantify phenotypic variation in promoters using plasmid system}
        
        As we have a comprehensive collection of MG1655 promoters cloned on plasmids upstream of a fluorescent reporter (GFP gene) available \cite{zaslaver2006comprehensive}, next step after the promoter selection is to transform selected MG1655 promoters on these plasmid systems into chosen environmental \tax{E. coli} strains.
        Transcriptional responses of those will be analysed under sets of conditions (e.g., different carbon sources for \tax{lacZ} promoter, exposure to sublethal concentrations of an antibiotic for \tax{recA} promoter) using flow cytometry.
        This will be done to determine whether there are differential expression means, variance levels and dynamics of promoters originating from MG1655 among environmental strains (i.e., exploring trans-effects).
        Further, we will also clone promoters originating from chosen environmental strains upstream of the fluorescent reporter into the same plasmid system.
        These \tax{promoter::GFP} systems will be cloned into their source strains and MG1655 and tested under appropriate sets of conditions as well.
        A subset of studied strains will also be picked to compare patterns acquired from flow cytometry under dynamic conditions.
        Fluorescent microscopy and microfluidics will be used for this.
        Besides studying the transcriptional responses only, microfluidics will allow us to look into how the activity of selected promoters reacts in a fluctuating environment and thus investigate the occurrence of epigenetic memory in those transcriptional networks.
        
        \item \textbf{Confirm observed relationships between genotypic and phenotypic variation by chromosomal integration}
        
        Plasmid-based models are relatively fast and easy to acquire, but they might cause problems in interpreting the results.
         For example, there is usually a single copy of promoter present in bacterial chromosome, while there are often multiple copies of a plasmid in a cell.
        There might also be different transcriptional activities on a plasmid compared to a chromosome, e.g., depending on NAPs binding.
        To check whether this is the case in studied promoters, we will clone a subset of promoters showing differential expression upstream of a fluorescent reporter and then integrate them into the chromosome of studied strains.
        We will then monitor variation under the same sets of conditions as for plasmid-based models (e.g., different carbon sources or sublethal concentration of antibiotic).
        Flow cytometry will be used for this.
        If different expression patterns are observed when compared to plasmid models, we will continue studying the expression of such promoters after chromosomal integration only.
        Thus fluorescent microscopy and microfluidics will be performed on a subset of strains having a promoter with fluorescent reporter integrated into a chromosome and showing various expression dynamics among promoters under flow cytometry.

    \end{enumerate}
    
    \item \textbf{Compare natural variation in transcriptional responses with neutral model}
    
    Results from the previous aim will give us some view on how much natural variation there is in expression, sensitivity, plasticity, and noise and also whether epigenetics plays a role in any of the selected transcriptional networks.
    However, to understand whether natural selection has acted to increase or decrease variation in such transcriptional traits, we need a neutral model to compare to.
    Thus, we will generate sets of promoters with random mutations - i.e., variation expected to occur if there is no selection acting on neither of the traits of chosen promoters.
    Such an approach might also help us determine whether there is natural selection acting on memory.
    This will be achieved by random mutagenesis and analysis of variation produced this way as described below.

    \begin{enumerate}[font=\bfseries]
    
        \item \textbf{Generate libraries of promoters by random mutagenesis}
        
        First, we need to create random variation in the selected promoters.
        Error-prone polymerases will be used for that.
        We will aim for promoters with one to several SNPs within the promoter sequence.
        A library of the variants will be then cloned upstream of a fluorescent reporter into a plasmid or further incorporated into the chromosome of all studied genotypic backgrounds.
        Chromosomal integration will be chosen over a plasmid-based model if the previous results show differences in expression between these two approaches.

        \item \textbf{Quantify phenotypic variation in promoters created by random mutagenesis}
        
        Next, we will measure the expression mean, variance and dynamics of randomly generated promoters in our environmental strains using fluorescent reporter assays (plasmid system or chromosomal integration) and flow cytometry.
        The same sets of conditions that were used for native promoters will be applied here (i.e., different carbon sources for \tax{placZ}, exposure to a various sublethal concentration of antibiotics for \tax{precA}).
        The variance in these values will be then compared to variance obtained in native promoters among selected strains.
        Based on the results we will be able to define whether the selection has acted to decrease or increase certain traits (i.e., expression level, noise, sensitivity, plasticity).
        In the cases when memory is observed in naturally occurring promoter variants, we will also check whether it is still present when appropriate promoters after the random mutagenesis are used.
        For this, a subset of strains with adequate promoters from the neutral model will be picked and analysed under fluctuating conditions using microfluidics.
    
    \end{enumerate}

\end{enumerate}

\cleardoublepage%%% keeps correct headings

\shorthandon{-} 
%%%%%%%%%%%%%%%%%%%%%%%%%%%%%%%%%%%%
