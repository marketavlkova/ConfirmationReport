\chapter{Objectives}

\shorthandoff{-} 

Thanks to the extensive collection of environmental \tax{E. coli} isolates \cite{ishii2006presence}, we have access to various strains originating from either the same collection area or different ones.
We will use these strains in order to elucidate how natural selection acts on transcriptional regulatory networks by investigating both genotypic and phenotypic variation in several promoters.

\section{Explore genotypic and phenotypic variation in promoters among environmental \tax{E. coli} isolates.}
%%% I'M NOT SURE HOW TO SPECIFY WHERE THE PRELIMINARY DATA WERE COLLECTED, SO I'M OPENED TO SUGGESTIONS
Some preliminary data were already collected in my supervisor's previous laboratory using \tax{lacZ} promoter.
They identified differences in the promoter sequence and further downstream in \tax{lacZ} gene itself in natural isolate SC1\textunderscore D9 when compared to classic laboratory strain MG1655.
Differential fluorescence was observed when these promoters were cloned upstream GFP gene and induced by IPTG.
And not only the sequence itself, but also genetic background seems to affect the fluorescence levels and thus \tax{lacZ} promoter activity.
I will follow up on their findings with \tax{lacZ} promoter and include other promoters as well.
The specific steps of this objective are described further below.

\subsection{Identify promoter sequences among environmental \tax{E. coli} isolates to study}
a

\subsection{Investigate what effect have the polymorphisms on the phenotypes}
a




\section{Compare natural variation in transcriptional responses with variation created by random mutagenesis.}
a

\subsection{Generate libraries of promoters by random mutagenesis}
a

\subsection{Investigate phenotypic variation in promoters created by random mutagenesis}
a



\shorthandon{-} 
%%%%%%%%%%%%%%%%%%%%%%%%%%%%%%%%%%%%
