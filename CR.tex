% \documentclass[11pt,a4paper,oneside,final]{book} %% FOR ONE-SIDE PRINT
% \documentclass[11pt,a4paper,twoside,final]{book} %% FOR TWO-SIDE PRINT
 \documentclass[12pt,a4paper,oneside,final]{book} %% FOR ONE-SIDE PRINT
% \documentclass[12pt,a4paper,twoside,final]{book} %% FOR TWO-SIDE PRINT

%%%%%%%%%%%%
%% Choose document coding system
%%%%%%%%%%%%

\usepackage[utf8]{inputenc} 
% \usepackage[cp1250]{inputenc} %% SETTINGS FOR WINDOWS
\usepackage[IL2]{fontenc} %fonts for Czech/Slovak, shouldn't make problems in English

\usepackage[german, czech]{babel}
% \usepackage[slovak]{babel}
% \usepackage[english]{babel}
% \usepackage{czech}
% \usepackage{slovak}

\usepackage{mathptmx}  %% free font Adobe Times Roman


%%%%%%%%%%%%%%%%%%%%%%%%%%%%%%%%%%%%%%%%%%%%%%%%%%%%%%%%%%%%%%%%%%%%%%%%%%%%%%%%%%%%%%%%%%%%%%%%%%%%
%%%%%%%%%%%%%%%%%%%%%%%%%%%%%%%% PACKAGES NECESSARY FOR THE TEMPLATE %%%%%%%%%%%%%%%%%%%%%%%%%%%%%%%%%%%%%%%
\usepackage[authoryear]{natbib}
%\PassOptionsToPackage{
 %       natbib=true,
	%			style=apa,
   %     url=true,
		%		hyperref=true,
     %       }   {biblatex}
\bibpunct{(}{)}{,}{a}{}{,}
\let\cite\citep

\usepackage{longtable}

%%%%%%%%%%%%%%%%%%%%%%%%%%%%%%%%%%%%%%%%%%%%%%%%%%%%%%%%%%%%%%%%%%%%%%%%%%%%%%%%%%%%%%%%%%%%%%%%%%%%
%%%%%%%%%%%%%%%%%%%%%%%%%%%%% PACKAGES FOR MATH %%%%%%%%%%%%%%%%%%%%%%%%%%%%%%%%%%%%%%%

\usepackage{amsmath,amssymb,amsthm}

\usepackage{floatrow} %%%capposition=top puts descriptions of tables above the tables

%%%%%%%%%%%%
%% SET THESE DETAILS
%%%%%%%%%%%%

\usepackage[ConfRep,Color]{mu.thesis}
%% Possible options:
%% Bc - for Bachelor's Thesis
%% Mgr - for Master's Thesis
%% Hons - for Honors Thesis
%% PhD - for Dissertation
%% ConfRep - for Confirmation Report
%% Color - for document with colored links
%% Tisk - for document in black & white

\DepartmentName{Institute of Natural and Mathematical Sciences}

\YearOfSubmit{2018}

\Author{Markéta Vlková}{Mgr. Markéta Vlková}

\ThesisName{Natural variation among \textit{E.~coli} isolates in transcriptional responses and memory}{{Natural variation among \textit{E.~coli} isolates in transcriptional}\\ & {responses and memory}}

\Supervisor{Dr. Olin Silander}

\Course{Microbiology and Genetics}

\PageNumber{?\,$+$\,?}

\Keys{\textit{E.~coli}; transcription; epigenetics}

\AbstractText%
{It is almost 160 years since Charles Darwin set out the theory of evolution by natural selection.
This theory is broadly accepted these days however, it is still not completely understood how does natural selection shape particular cell mechanisms and behaviours.
This work aims to get a better view on the role natural selection plays in transcriptional responses and memory in five transcriptional networks of a single species.

This will be done by comparing natural genotypic and phenotypic variation in promoters to a neutral model.
We will use fluorescent reporters together with flow cytometry and fluorescent microscopy to quantify changes in expression levels and dynamics (i.e. noise, plasticity, speed and sensitivity).
To detect epigenetic memory and transcriptional responses under fluctuating conditions microfluidics will be performed.
Studied sequences will be cloned into plasmid based model, but chromosomal integration will take place as well.

For the beginning, well studied \textit{lacZ} and \textit{recA} promoters were used to establish the main workflows.
Preliminary results show that both the sequence and genetic background have effect on the promoter activity (i.e. cis- and trans- effect).
A new plasmid for chromosomal integration was also constructed in order to check  and correct for possible aberrations caused by plasmid based models.

This study brings a new way for studying natural selection, as previous works focused mainly on lab strains.
Here we use a collection of environmental \textit{E.~coli} isolates to look into how natural selection has acted really in the nature.
}

\AknowledgementText%
{}

%%%%%%%%%%%%%%%%%%%%%%%%%%%%%%%%%%%%%%%%%%%%%%%%%%%%%%%%%%%%%%%%%%%%%%%%%%%%%%%%%%%%%%%%%%%%%%%%%%%%%%
%%%%%%%%%%%%%%%%%%%%%%%%%%%%%%%%%%%% YOUR OWN COMMANDS %%%%%%%%%%%%%%%%%%%%%%%%%%%%%%%%%%%%%%%%%%%%%%%%
%%%%%%%%%%%%%% HERE YOU CAN DEFINE YOUR OWN COMMANDS TO MAKE YOUR WRITTING EASIER %%%%%%%%%%%%%%%%

\newcommand{\Cbb}{\mathbb{C}}
\newcommand{\Rbb}{\mathbb{R}}
\newcommand{\Zbb}{\mathbb{Z}}
\newcommand{\Nbb}{\mathbb{N}}

%\usepackage[draft]{graphicx} %%%this was in the original version, but i got problem with it and don't know why, so I disabled it%%%


%%%%%%%%%%%%%%%%%%%%%%%%%%%%%%%%%%%%%%%%%%%%%%%%%%%%%%%%%%%%%%%%%%%%%%%%%%%%%%%%%%%%%%%%%%%%%%%%%%%%%%
%%%%%%%%%%%%%%%%%%%%%%%% CREATE AUXILIARY FILE FOR INDEX %%%%%%%%%%%%%%%%%%%%%%%%%%%%%%%%%%%%%%%%%%
\makeindex

\usepackage{titlesec}
\usepackage{multirow}
\usepackage{tabularx}
\usepackage{booktabs}
\usepackage{longtable}
\setlength{\LTcapwidth}{8in} %%% nastaví šířku popisu tabulek "longtable" - normálně nejsou stejně široké jako u "table" %%%
\usepackage{pdfpages} %%% pro možnost vložení externího pdf souboru %%%
\usepackage{hyperref} %%% pro crosslink příloh v pdf %%%
\newcounter{includepdfpage} %%% pro zprovoznění crosslinku u příloh %%%
%\usepackage[usenames, dvipsnames]{color} %%% pro definování vlastních barev písma %%%
\usepackage{enumitem}  %%% pro redukci mezer v itemize %%%
\usepackage{multicol} %%% pro itemize ve víze sloupcích %%%

%%%%%%%%%%%%%%%%%%%%%%%%%%%%%%%%%%%%%%%%%%%%%%%%%%%%%%%%%%%%%%%%%%%%%%%%%%%%%%%%%%%%%%%%%%%%%%%%%%%%%%
%%%%%%%%%%%%%%%%%%%%%%%%%%%%%%%%%% DOCUMENT BEGINNING %%%%%%%%%%%%%%%%%%%%%%%%%%%%%%%%%%%%%%%%%%%%%%%%%%
%%%%%%%%%%%%%%%%%%%%%%%%%%%%%%%%%%%%%%%%%%%%%%%%%%%%%%%%%%%%%%%%%%%%%%%%%%%%%%%%%%%%%%%%%%%%%%%%%%%%%%

\begin{document}

%%%%%%%%%%%%%%%%%%%%%%%%%%%%%%%%%%%%%%%%%%%%
%%%%%%%%%%%% OUTSET %%%%%%%%%%%%%%%%%

\MakeOutset

%%%%%%%%%%%%%%%%%%%%%%%%%%%%%%%%%%%%%%%%%%%%%%%%%%%%%%%%%%%%%%%%%%%%%%%%%%%%%%%%%%%%%%%%%%%%%%%%%%%
%%%%%%%%%%%%%%%%%%%%%%%%%%%%%%%%%% ABSTRACT %%%%%%%%%%%%%%%%%%%%%%%%%%%%%%%%%%%%%%%%%%%%

\AbstractPage

%%%%%%%%%%%%%%%%%%%%%%%%%%%%%%%%%%%%%%%%%%%%%%%%%%
%%%%%%%%%% AKNOWLEDGEMENT %%%%%%%%%%%%%%%

%\AknowledgementPage


%%%%%%%%%%%%%%%%%%%%%%%%%%%%%%%%%%%%%%%%%%%%%%%%%%
%%%%%%%%%%%%% POUZE PROHLASENI %%%%%%%%%%%%%%%%%%%
%
% \ProhlaseniBezPodekovani
%
%%%%%%%%%%%%%%%%%%%%%%%%%%%%%%%%%%%%%%%%%%%%%%%%%%
%%%%%%%%%%%%%%%%%%% OBSAH %%%%%%%%%%%%%%%%%%%%%%%%

\renewcommand{\contentsname}{Table of Contents}
\pdfbookmark{Table of Contents}{Table of Contents}
\setcounter{tocdepth}{4}	%%% to include subsubsections in table of content
\MakeContent
\addtocontents{toc}{~\hfill\textbf{Page}\par}%%% adds word "Page" above the list of page numbers in the table of contents
\cleardoublepage

%%%%%%%%%%%%%%%%%%%%%%%%%%%%%%%%%%%%%%%%%%%%%%%%%%%%%%%%%%%%%%%%%%%%%%%%%%%%%%%%%%%%%%%%%%%%%%%%%%%
%%%%%%%%%%%%%%%%%%%%%%%%%%%%%%%%%%% TEXT PRACE %%%%%%%%%%%%%%%%%%%%%%%%%%%%%%%%%%%%%%%%%%%%%%%%%%%%

\renewcommand{\chaptername}{Chapter}
\renewcommand{\chaptermark}[1]{\markboth{\thechapter. #1}{}}
\renewcommand{\sectionmark}[1]{\markright{\thesection. #1}{}}
\renewcommand{\figurename}{\underline{Figure}}
\renewcommand{\tablename}{\underline{Table}}
\floatsetup[table]{capposition=top}

\newcommand{\tax}[1]{\mbox{%
	\textit{%
		#1%
	}%
}}

\HeadingAbbrev
\chapter*{List of Abbreviations}
\addcontentsline{toc}{chapter}{Abbreviations}
\renewcommand{\chaptername}{Abbreviations}

Here I present an alphabetical list of used abbreviations for easier orientation in the following text.
\begin{flushleft}
\begin{longtable}[l]{ll} %% [l] tabulka je zarovnana vlevo; [c] zarovnani na stred; [r] zarovnani v pravo
	EBP		& enhancer binding protein \\[1mm]
	EPEC	& enteropathogenic \tax{Escherichia coli} \\[1mm]
	FFL		& feed-forward loop \\[1mm]
	HK		& histidine kinase \\[1mm]
	IPTG		& isopropyl-$\beta$-D-thiogalactoside \\[1mm]
	MT		& methyltransferase \\[1mm]
	NAP		& nucleoid-associated protein \\[1mm]
	PA		& protein aggregate \\[1mm]
	PBS		& phosphate-buffered saline \\[1mm]
	PCR		& polymerase chain reaction \\[1mm]
	RR		& response regulator \\[1mm]
	STEC	& Shiga-toxin producing \tax{Escherichia coli} \\[1mm]
	TMG		& thiomethyl-$\beta$-D-galactoside \\[1mm]
	TSS		& transformation and storage solution \\[1mm]
	UPEC	& uropathogenic \tax{Escherichia coli}
\end{longtable}
\end{flushleft}

\cleardoublepage


\HeadingChapters
\renewcommand{\chaptername}{Introduction}
\chapter*{Introduction}
\setcounter{page}{1}
\pagenumbering{arabic}
\addcontentsline{toc}{chapter}{Introduction}


\shorthandoff{-} 

\section{Taxonomie rodu \tax{Pseudomonas}}
%\addcontentsline{toc}{section}{Taxonomie rodu \tax{Pseudomonas}}
               Doména:	\hspace{0,5cm} \tax{Bacteria}\\%
\hspace*{1,5cm} Kmen:	  \hspace{0,5cm} \tax{Proteobacteria}\\%
\hspace*{2,5cm} Třída:   \hspace{0,5cm} \tax{Gammaproteobacteria}\\%
\hspace*{3,5cm} Řád:     \hspace{0,5cm} \tax{Pseudomonadales}\\%
\hspace*{4,5cm} Čeleď:   \hspace{0,5cm} \tax{Pseudomonadaceae}\\%

Během posledních let došlo k výrazné změně taxonomie tohoto rodu, současně se~změnou bakteriální taxonomie jako takové.
Na základě nového přístupu členění bakterií do~rodů dle příbuznosti genu pro 16S rRNA bylo mnoho zástupců řazených pod \tax{Pseudomonas}~spp. reklasifikováno.
I přes to, že druhů z původního souboru příliš nezůstalo, tento rod patří s~více než dvěmi sty taxony k druhově nejbohatším \cite{www.bacterio.net}.
Zásluhu na tom má stále rostoucí počet nově popsaných zástupců. \cite{cornelis2008pseudomonas}

\subsection{Historie a současnost}
Objev a definice prvních pseudomonád se datuje do konce 19. století.
Následovalo velké množství nově popsaných druhů izolovaných z různých zdrojů.
V polovině 20.~století bylo k tomuto rodu řazeno odhadem přes 800 druhů. \cite{palleroni2010pseudomonas}
Páté vydání Bergeyho manuálu z roku 1957 již zahrnovalo tyto zástupce a jejich fenotypové vlastnosti.
Bohužel obsažené informace byly čerpány především z amerických studií a postrádaly několik důležitých zjištění, ke kterým přišli evropští badatelé.
Tak se například americká bakteriologická komunita dozvěděla o bohaté metabolické aktivitě pseudomonád až v 60.~letech, ačkoliv první práce popisující tyto charakteristiky je z roku 1926. \cite{palleroni2010pseudomonas}

Morfologie pseudomonád se příliš neliší od jiných i velmi fylogeneticky vzdálených rodů.
Tudíž vlastnosti jako tvar, velikost, uspořádání buněk, přítomnost inkluzí a pohyblivost byly doplňovány množstvím fyziologických testů - produkcí pigmentů, růstovými, nutričními a teplotními podmínkami, akumulací biopolymerů či testy produkce intra i extracelulárních enzymů.
Taxonomicky užitečná se ukázala schopnost produkce fluorescentních sideroforů.
Popis druhů byl posléze doplněn daty ohledně procentuálního zastoupení bází. \cite{cornelis2008pseudomonas, palleroni2010pseudomonas}

V 70. letech byla publikována práce, která velmi zredukovala dosavadní počet pseudomonád a následně ovlivnila i celou taxonomii bakterií. \cite{palleroni1973nucleic, frey1997molecular, clarridge2004impact, larsen2014benchmarking}
Bylo zjištěno, že zástupci zahrnovaní do rodu \tax{Pseudomonas} (nyní \tax{Pseudomonas sensu lato}), se dělí do pěti ostře ohraničených skupin dle podobnosti rRNA.
Jako \tax{Pseudomonas sensu stricto} pak byla vyčleněna rRNA skupina I, která obsahovala typový druh \tax{P. aeruginosa}, všechny fluorescentní a některé nefluorescentní druhy.
Navazující studie na jiných bakteriích prokázaly, že konzervovanost rRNA je velmi užitečná v rodové diferenciaci a taxonomie do té doby postavená především na morfologických a fyziologických znacích byla přehodnocena. \cite{ fox1977comparative, woese1980phylogenetic, ludwig1981phylogenetic}
Dnes nám již pro popis nových druhů nestačí pouze fenotypová charakteristika, ale je třeba přistoupit k tzv. polyfázové taxonomii. \cite{tindall2010notes, thompson2015microbial}
Fenotypovou charakteristiku zpravidla doplňují molekulární analýzy a chemotaxonomie, dokonce se jim povětšinou věnuje více pozornosti než klasické fenotypizaci.
Avšak ani bez jednoho přístupu se v dnešní komplexní taxonomii neobejdeme. \cite{palleroni1973nucleic, palleroni2010pseudomonas}

\section{Charakteristika rodu \tax{Pseudomonas}}
Typovým rodem čeledi \tax{Pseudomonadaceae} je \tax{Pseudomonas} Migula 1894.
Jedná se o~rovné či lehce zakřivené gramnegativní tyčinky, které netvoří klidová stádia a nehromadí granula poly-$\beta$-hydroxybutyrátu.
Zástupci tohoto rodu jsou povětšinou opatřeni jedním či~více polárními bičíky.
Mají striktně respiratorní metabolismus s molekulárním kyslíkem jako konečným akceptorem elektronů.
Nitrát však může v některých případech sloužit jako alternativní akceptor elektronů a umožňovat tak růst za mírně anaerobních podmínek.
Tito aerobní chemoorganotrofové obecně nevyžadují růstové faktory a charakteristická je pro~ně produkce katalázy, přičemž oxidázová aktivita nemusí být přítomna.
Typovým druhem tohoto rodu je \tax{Pseudomonas aeruginosa}. \cite{garrity2005bergey}

\subsection{Bičíky a fimbrie}
%\addcontentsline{toc}{subsection}{Bičíky a fimbrie}
Ačkoliv je pro pseudomonády typické polární uložení bičíků, v některých případech se mohou vyskytovat subpolárně či laterálně (\tax{P. stutzeri}, \tax{P. mendocina}).
Zato počet bičíků (resp. přítomnost právě jednoho či více než jednoho) je důležitým taxonomickým znakem.
U polárního monotricha \tax{P. aeruginosa} byly objeveny na lokusu fliC dva geny kódující flagelin, přičemž jeden z nich je variabilní. \cite{spangenberg1996genetic}

Přítomnost polárních fimbrií byla detekována u \tax{P. aeruginosa} a \tax{P. alcaligenes}.
Druhy \tax{P. fluorescens}, \tax{P. chlororaphis} a \tax{P. putida} naopak fimbrie (pili) postrádají.
Tyto struktury se podílejí např. na adhezi či trhavém pohybu buněk po pevném povrchu, ale mohou sloužit i jako receptory pro bakteriofágy. \cite{garrity2005bergey}



\cleardoublepage



\HeadingObjectives
\chapter*{Objectives}
\addcontentsline{toc}{chapter}{Objectives}

\shorthandoff{-} 

Thanks to the collection of environmental \tax{E. coli} strain isolated near Lake Superior, MN, USA \cite{ishii2006presence}, we have access to 480 various strains thorough the whole phylogenetic tree of this species.
We will use these strains in order to elucidate how natural selection acts on transcriptional responses and epigenetic memory of several promoters by investigating both genotypic and phenotypic variation in them.

\begin{enumerate}[font=\bfseries]

	\item \textbf{Quantify genotypic and phenotypic variation in promoters among environmental \tax{E. coli} isolates}
	
	In order to quantify variation in transcriptional, we will focus initially on studying natural variation in the response of the \tax{lac} operon in the presence of different carbon sources, either separately or in combination.
	To begin we will study these responses using a promoter-based system.
	Previously, initial data were collected which suggested that differences in the \tax{lac} promoter sequence and further downstream in \tax{lacZ} gene itself result in differential gene expression.
	Moreover not only the sequence of the promoter, but also the genetic background of the strain was observed to affect the level of expression (i.e. cis- and trans- effects).
	I will follow up on and expand these findings with \tax{lacZ} promoter and include other promoters and strains as well.
	The specific steps of this objective are described further below.

	\begin{enumerate}[font=\bfseries]
	
		\item \textbf{Choose promoter sequences among environmental \tax{E. coli} isolates to study}
		
		For the beginning we have selected the well studied \tax{lacZ} and \tax{recA} promoters, because each create a different type of cell response - carbon source catabolism and stress response, respectively.
		We will select three(\textcolor{red}{?}) other transcriptional circuits to study as well.
		Promoters with known function and regulation mechanism will be picked based on annotated MG1655 genome.
		Then we will search for corresponding sequences in genomes of natural isolates and identify the sequence differences in them.
		As we want to cover a broad variation scale on genetic level we will select promoter sequences which are both highly and slightly diverse among natural isolates.
		Those which are mainly bound by RNA polymerase with a conventional $\sigma^{70}$ factor will be preferred.

		\item \textbf{Quantify phenotypic variation in promoters using plasmid system}
		
		% 17.9.2018
		% maybe let's say as a first step to look for differences in responses we will study the plasmid system using the (already clones) K12 sequence. I think this makes sense.


		\item \textbf{Confirm observed relationships between genotypic and phenotypic variation by chromosomal integration}
		
		As plasmid based models might cause problems in interpreting the results in respect to usually single copy of promoter in bacterial chromosome and multiple copies of plasmid, we will clone a subset of promoters showing differential expression upstream of a fluorescent reporter and then integrate them into chromosome of studied strains.
		We will then monitor variation under sets of conditions to determine the dynamics in studied transcriptional responses.
		For this purposes flow cytometry will be mainly used.
		However, we want to investigate these trait not only at single cell level, but under dynamic conditions as well, so fluorescent microscopy and microfluidics with high throughput image data analysis will follow.
		Microfluidics can also help us to go deeper into transcriptional regulation under fluctuating environments and potential epigenetic memory.

	\end{enumerate}
	
	\item \textbf{Compare natural variation in transcriptional responses with random distribution}
	
	Results from the previous aim will give us some view on how much natural variation there is in expression, sensitivity, plasticity and noise.
	To understand whether selection has acted to increase or decrease such variation, we need a neutral model to compare to.
	Thus, we will generate sets of promoters with random mutations - i.e. variation expected to occur if there is no selection acting on the particular trait of promoter.
	This will be achieved by a random mutagenesis and analysis of variation produced this way as delineated below.

	\begin{enumerate}[font=\bfseries]
	
		\item \textbf{Generate libraries of promoters by random mutagenesis}
		
		First, we need to create a random variation in the selected promoters.
		% 17.9.2018
		% perhaps a bit more detail, e.g. avg number of changes per promoter, number of variants (will you study 10 or 100 or 1000)...
		Error-prone polymerases will be used for that and produced variants will be then cloned upstream of a fluorescent reporters and incorporated into chromosome of all studied genotypic backgrounds the same way as for the naturally occurring promoters.

		\item \textbf{Investigate phenotypic variation in promoters created by random mutagenesis}
		
		% 17.9.2018
		% "same techniques as previously" - again more detail. I guess you can study hundreds using cytometry but only select a subset to study with microscopy. You could say this.
		Next, we will compare the observed phenotypic variation in native promoters to those randomly generated among studied strains using the same techniques as previously.
		% 17.9.2018
		% be more direct: "whether selection has acted to decrease or increase..."
		Based on those results we will be able to define in which direction the selection acts on particular promoters compared to the random distribution.
		We should be also able to answer a question whether the chosen promoters were selected for a certain level of sensitivity, noise or memory and if so, which one.

	
	\end{enumerate}

\end{enumerate}


\shorthandon{-} 
%%%%%%%%%%%%%%%%%%%%%%%%%%%%%%%%%%%%


\HeadingChapters
\chapter{Materials and Methods}
%\setcounter{page}{1}
%\pagenumbering{arabic}
%\addcontentsline{toc}{chapter}{Metody}


%%%%%%%%%%%%%%%%%%%%%%%%%%%%%%%%%%%%
%%%%%%%%% GENERUJ TEXT %%%%%%%%%%%%%

\shorthandoff{-} 

\section{Strains and growth conditions}
All strains used in this work belong to the genus \tax{Escherichia coli} and are derived from a classical laboratory strain K12 (MG1655) and an environmental strain SC1\textunderscore D9 isolated from a bank of St. Louis River in Minnesota within a great sampling study \cite{ishii2006presence}.
In future work additional environmental isolates are likely to be used.

\begin{center}
	\begin{longtable}[c]{|l|c|c|c|c|}
\caption{Strains used in the study so far} \label{strains} \\

\toprule \multicolumn{1}{|l|}{\textbf{Strain ID}} & \multicolumn{1}{c|}{\textbf{Relevant genotype}} & \multicolumn{1}{c|}{\textbf{Promoter}} & \multicolumn{1}{c|}{\textbf{Ancestor}} & \multicolumn{1}{c|}{\textbf{Reference}} \\
\midrule
\endhead

\bottomrule
\endlastfoot

MG1655 & \tax{E. coli} K12 wild-type & - & - & \cite{blattner1997complete} \\
\hline
SC1\textunderscore D9 & ? & - & - & ? \\
%%% should I use the same reference as above? (ishii2006presence)
\hline
pUA66-K12 & pUA66 & - & MG1655 & \cite{zaslaver2006comprehensive} \\
\hline
pU139-K12 & pU139 & - & MG1655 & \cite{zaslaver2006comprehensive} \\
\hline
pU139-D9 & pU139 & - & SC1\textunderscore D9 & this study \\
\hline
ASC662 & \tax{lacZ-GFP} & - & MG1655 & \cite{kiviet2014stochasticity} \\
\hline
pZ1\textunderscore K12-K12 & \tax{placZ::GFP} & MG1655 & MG1655 & \cite{zaslaver2006comprehensive} \\
\hline
pZ2\textunderscore K12-K12 & \tax{placZ::GFP} & MG1655 & MG1655 & this study \\
\hline
pZ2\textunderscore K12-D9 & \tax{placZ::GFP} & MG1655 & SC1\textunderscore D9 & ? \\
\hline
pZ\textunderscore D9-D9 & \tax{placZ::GFP} & SC1\textunderscore D9 & SC1\textunderscore D9 & ? \\
\hline
pZ\textunderscore D9-K12 & \tax{placZ::GFP} & SC1\textunderscore D9 & MG1655 & ? \\
\hline
pZm1\textunderscore D9-K12 & \tax{placZm172::GFP} & SC1\textunderscore D9 & MG1655 & ? \\
\hline
pZm2\textunderscore D9-K12 & \tax{placZm279::GFP} & SC1\textunderscore D9 & MG1655 & ? \\
\hline
pA\textunderscore K12-K12 & \tax{precA::GFP} & MG1655 & MG1655 & \cite{zaslaver2006comprehensive} \\
\hline
pA\textunderscore K12-D9 & \tax{precA::GFP} & MG1655 & SC1\textunderscore D9 & this study \\
\hline
Top10 & \text{*} & - & MG1655 & ? \\
\hline
pLW001-Top10 & pLW001 & - & Top10 & ? \\
	\end{longtable}
\footnotesize
	\emph{\text{*}} F– mcrA $\Delta$(mrr-hsdRMS-mcrBC) $\Phi$80lacZ$\Delta$M15 $\Delta$lacX74 recA1 araD139 $\Delta$(ara leu)7697 galU galK rpsL(Str$^{R}$) endA1 nupG\\*
%%% I took it from Quartzy and found out it differs a bit from E. coli genotypes description in https://openwetware.org/wiki/E._coli_genotypes#TOP10_.28Invitrogen.29
%%% is it correct???
\end{center}

For most purposes M9 minimal medium supplemented with 0.4\% of a single carbon source was used.
As those D(+)-glucose, lactose, D(+)-galactose, L-arabinose and D(-)-ribose were chosen.
For DNA isolation all strain were grown in LB medium (liquid or 1.5\% agar).
Kanamycin to concentration 50$\mu$g/ml was also added to the media if the bacterium contained pU139 or pUA66 plasmid (with or without promoter upstream of GFP).
All strains were grown at 37$^{\circ}$C, except for pLW001-Top10 which was grown at 30$^{\circ}$C.

\section{Uchovávání izolátů}
Čisté izolované kmeny i použité referenční kmeny (CCM) byly uchovávány v hlubokomrazících boxech při -70$^{\circ}$C na keramických nosičích (korálcích) v polypropylenových zkumavkách.

Testované antarktické kmeny jsem z hlubokomrazících boxů oživovala přenesením jednoho korálku vyžíhanou kličkou do 0,5 ml bujónu ve zkumavce se šikmým R2A agarem [Oxoid].
Bujón se následně nechal stéci po celé ploše agaru a zkumavky se daly inkubovat na~2~až~7~dnů do~15$^{\circ}$C.

\section{Fenotypové metody}
\subsection{Morfologie buněk a kolonií}
Popis morfologie buněk, Gramovo barvení, zapsání tvaru a barvy kolonií proběhlo dle~běžných standardů České sbírky mikroorganismů.

\subsection{Biochemické testy}
Základních 34 testů u všech gramnegativních nefermentujících kmenů z Antarktidy bylo provedeno dle standardních postupů České sbírky mikroorganismů.
Na základě těchto výsledků se vytypovali presumptivní zástupci pseudomonád, kteří byli dále konfirmováni pomocí uniplex PCR.

Dodatečné testy k detekci exprese enzymů a využívání dalších látek jsem prováděla u~240 kmenů, jenž byly vyhodnoceny dle uniplex PCR jako \tax{Pseudomonas} sp.
Pro stanovení přítomnosti či absence daného znaku jsem použila klasické zkumavkové a miskové testy.
Kultivace probíhala v 15$^{\circ}$C či 30$^{\circ}$C (dle optimálního růstu jednotlivých kmenů).

\subsubsection{Okyselování a alkalizace cukrů}
Testovala jsem schopnost jednotlivých kmenů využívat L-arabinózu, D-arabinózu, ribózu, galaktózu, trehalózu a sacharózu jako zdroj uhlíku a energie (acidifikace media).

Polotuhé medium jsem připravila přidáním 10~ml zásobního 10\% cukerného roztoku předem vysterilizovaného při 118$^{\circ}$C po 20 minut do 90~ml sterilního OF bazálního media [Merck] (11~g/l; sterilizace: 121$^{\circ}$C, 15~min.) ochlazeného na cca 60$^{\circ}$C.
Po důkladném promíchání jsem roztok ještě za tepla rozplnila do sterilních zkumavek.

Bakteriální kmeny jsem očkovala vpichem.
Výsledky se odečítaly po 1 až 8 dnech kultivace.
Tvorba kyselin se projevila změnou barvy původního neutrálního media ze~zelené na~žlutou.
V případě alkalizace se medium zbarvilo do modra.
Jako pH indikátor zde sloužila bromtymolová modř.

\subsubsection{Hydrolýza elastinu}
Jde o test na detekci produkce extracelulárního enzymu elastázy, jenž může být znakem virulence.

Použité medium se sestávalo z 3~g elastinu s kongočervení a 20~g agaru [HiMedia] na 1~litr BHI (brain-heart infusion) [Bio-Rad].
Po 15 minutové sterilizaci při 121$^{\circ}$C jsem medium rozlila do Petriho misek.

Na misku bylo pomocí bakteriologické kličky očkováno po osmi kmenech.
Odečet výsledků probíhal po 1 až 14 dnech kultivace.
Hydrolýza elastinu se projevila projasněním media v místě růstu bakterie příp. i v jejím okolí.
Zaznamenávala jsem též černání media v~místě nárůstu kmenů.

\shorthandon{-} 
%%%%%%%%%%%%%%%%%%%%%%%%%%%%%%%%%%%%


\chapter{Results and discussion}

\shorthandoff{-} 

\section{Cloned \tax{promoter::GFP} sequences}
As most of the strains listed in the Table \ref{strains} were prepared several years ago and not by myself, we wanted to check which parts of the promoters we actually have cloned upstream of GFP gene in those cells.
For this purpose I designed pUA66\textunderscore insert primers (Table \ref{pcr}) based on the pUA66 and pU139 vector sequences (\hyperlink{pUA66seq}{Appendix 1} and \hyperlink{pUA66seq}{Appendix 2}).
PCR products were sequenced by Macrogen, Korea using the Sanger platform.

\hypertarget{SeqRes}{\subsection{\tax{placZ::GFP} strains}}
In Fig. \ref{placZ} we can see that SC1\textunderscore D9 and MG1655 differ in \tax{lacZ} promoter region by a single SNP located 39 nt upstream from the \tax{lacZ} gene start codon (position 199 in Fig. \ref{placZ}).
These 2 strains have 3 more SNPs further downstream in \tax{lacZ} gene itself 
(positions 306, 324 and 363 in Fig. \ref{placZ}).
Based on the alignment of acquired sequencing results (Fig. \ref{placZ}) we can confirm that \tax{lacZ} promoters assumed to originate from SC1\textunderscore D9 have an appropriate sequence as well as for MG1655.
\begin{figure}[b]
  \centering
  \includegraphics[scale=0.25]{text/Pictures/placZsequences.png}
    \caption{Excerpt of \tax{lacZ} promoters alignment with vectors and references. SC1\textunderscore D9 and MG1655 are reference sequences of \tax{lacZ} promoters from particular strains. pUA66 and pU139 are promoter-less vectors.}
    \label{placZ}
\end{figure}
Beside the SNPs in \tax{lacZ} promoter, the first SNP from \tax{lacZ} coding sequence is included in all vectors upstream of GFP.
Most of the strains possess pU139 as the vector except for pZ1\textunderscore K12-K12 which was acquired from Alon library \cite{zaslaver2006comprehensive}.
This strain also has a longer part of \tax{lacZ} gene cloned into its pUA66 vector, potentially covering the second SNP when compared to SC1\textunderscore D9.
This is not the only difference found in pZ1\textunderscore K12-K12, as part of the \tax{lacI} gene is incorporated in all sequences listed in Fig. \ref{placZ} on the opposite side of the promoter (\hyperlink{placZalign}{Appendix 3}) and in pZ1\textunderscore K12-K12 this \tax{lacI} sequence is almost 60 bp shorter than in the other strains.
Note that strain pZ1m1\textunderscore D9-K12 and pZ1m2\textunderscore D9-K12 possess a recurrent mutation of the SNP in \tax{lacZ} promoter or \tax{lacZ} gene, respectively (from SC1\textunderscore D9 back to MG1655).
Strain pZ2\textunderscore K12-K12 is not included in the Fig. \ref{placZ}.
The reason is that it was made by transformation of the plasmid from pZ2\textunderscore K12-D9 into MG1655 after receiving the sequencing data and was not sequenced afterwards.

\subsection{\tax{precA::GFP} strains}
Both pA\textunderscore K12-K12 and pA\textunderscore K12-D9 strains have the same plasmid (pU139) as the latter was made by transforming the plasmid from the former.
It was confirmed by sequencing as well as that the sequence upstream of the GFP gene corresponds to the \tax{recA} promoter of MG1655 (\hyperlink{precAalign}{Appendix 4}).
Similar to the cloned \tax{lacI}-\tax{lacZ} intergenic regions mentioned above, the sequence incorporated into pU139 and then transformed into these \tax{precA::GFP} strains includes parts of \tax{recA} gene downstream and \tax{pncC} gene upstream of the promoter.


\section{Ciprofloxacin susceptibility}
Because we want to induce the \tax{recA} promoter by exposing the cells to a sub-lethal (i.e., sub-MIC) concentration to Ciprofloxacin, I tested the susceptibility of MG1655 and SC1\textunderscore D9 strains to Ciprofloxacin.
I also wanted to estimate what are the minimum inhibitory concentrations (MICs) of this antibiotic for the strains used.
MG1655 and SC1\textunderscore D9 were incubated in an increasing concentration of Ciprofloxacin (see \hyperlink{MIC}{Materials and Methods}).
\begin{figure}[b!]
  \centering
  \includegraphics[scale=0.25]{text/Pictures/KillCurve.png}
    \caption{Cell density after 24 h growth of MG1655 and SC1\textunderscore D9 strains in various Ciprofloxacin concentrations. Strains with a promoter-less plasmid are also included.}
    \label{killing}
\end{figure}
Strains with a promoter-less pU139 or pUA66 plasmid were also included to ensure the Kanamycin resistance carried on the plasmid does not affect Ciprofloxacin sensitivity.
The MIC for both strains is 10 ng/ml of Ciprofloxacin or less; however, 5 ng/ml is a concentration which both strains are able to survive, although a decrease in the final cell density is observed when compared to the growth in the absence of Ciprofloxacin (Fig. \ref{killing}).
Plasmid carried Kanamycin resistance causes none or a small effect to Ciprofloxacin susceptibility.


\section{Variation in promoter activity}
To explore the differences in promoter activity and genetic backgrounds 4 h cultures grown in M9 minimal media with a single carbon source were assayed using flow cytometry (see \hyperlink{FC}{Materials and Methods}).
Exported FCS files were then analysed in R - the scripts used are available at \href{https://github.com/marketavlkova/}{GitHub} repositories.

\subsection{Activity of \tax{lacZ} promoters}
\phantomsection%%% removes warning: "package hyperref warning the anchor of a bookmark and its parent's must not be the same. Added a new anchor on input line 55."
\subsubsection{Expression in lactose environment}
All strains with \tax{lacZ} promoter cloned upstream of GFP gene and strain ASC662, which has GFP integrated into the chromosome in the \tax{lac} operon, show strong GFP expression when grown in lactose as a sole carbon source (Fig. \ref{lacZassay}).
All plasmid-based models show about 2.5 to 10.5 times higher GFP expression (3.842 au, stdev: 0.113 in pZ1\textunderscore K12-K12 to 4.464 au, stdev: 0.095 in pZm1\textunderscore D9-K12) than ASC662 (3.442 au, stdev: 0.140).
Expression differences are also observed when promoter sequences of the same length, but originating from different strains (2 SNPs present) are cloned into the same genotypic background.
Strain pZ\textunderscore D9-D9 has over 1.5 times higher mean GFP expression than pZ2\textunderscore K12-D9 (4.403 au, stdev: 0.202 and 4.219 au, stdev: 0.134, respectively).
Similarly, pZ\textunderscore D9-K12 has almost 1.5 times higher mean expression than pZ2\textunderscore K12-K12 (4.437 au, stdev: 0.117 and 4.267 au, stdev: 0.103, respectively).
Over 2.5-fold difference in expression is also observed when comparing both strains with two versions of cloned promoter sequence originating from MG1655 into the same strain, i.e., pZ1\textunderscore K12-K12 and pZ2\textunderscore K12-K12.
The former having mean expression 3.842 au, stdev: 0.113, the latter 4.267 au, stdev: 0.103.
We think this is caused by differences in 5' end of the produced mRNA between these two plasmids, which might have an effect on the mRNA stability or translation efficiency.

\begin{figure}[ht!]
  \centering
  \includegraphics[scale=0.4]{text/Pictures/lacZassay.png}
    \caption{\tax{lacZ} promoter activity in various carbon sources. Strain ASC662 has GFP gene integrated into chromosome in the native locus of \tax{lac} operon. pU139-K12 is a negative control of GFP expression for plasmid-based models - a promoter-less pU139 plasmid present. The rest are strains with various \tax{lacZ} promoters cloned upstream of the GFP gene in a plasmid - see Table \ref{strains}.}
    \label{lacZassay}
\end{figure}

The strains pZ\textunderscore D9-K12 and pZ2\textunderscore K12-K12 differ by just the two SNPs in the sequence cloned upstream of GFP gene, while one of the SNPs is located in the cloned part of \tax{lacZ} gene, not in the promoter itself (see \hyperlink{SeqRes}{sequencing results}).
If the SNPs from \tax{lacZ} gene had no effect on the GFP expression, we should see the same expression pattern in pZm1\textunderscore D9-K12 as in pZ2\textunderscore K12-K12 (i.e., 4.267 au, stdev: 0.103).
However, this is not the case - the recurrent mutation of the first SNP (A to G) leads to even higher mean expression difference (4.464 au, stdev: 0.095; i.e., 1.57-fold change) than is seen when the MG1655 promoter is compared to the original sequence from SC1\textunderscore D9 (4.437 au, stdev: 0.117; i.e., 1.48-fold change).
Interestingly we get much closer mean expression values to MG1655 promoter sequence with the recurrent mutation in the \tax{lacZ} coding sequence.
This time pZm2\textunderscore D9-K12 shows a bit lower expression level than pZ2\textunderscore K12-K12 (4.249~au, stdev: 0.105 vs. 4.267 au, stdev: 0.103; i.e., 1.04-fold change).
This might be caused by differential mRNA stability and/or translation efficiency in the plasmid-based models.
As the second SNP (located in \tax{lacZ} gene itself) is transcribed into 5' end of the mRNA coding GFP in our models it might affect the mRNA folding.

\subsubsection{Expression in non-lactose environments}
ASC662 does not express GFP when grown in any other carbon source used, as the fluorescence levels are the same as for negative control pU139-K12.
Negative control for SC1\textunderscore D9 background, i.e., pU139-D9 is comparable to pU139-K12 (\hyperlink{FCnegs}{Appendix 5}).
Interestingly some \tax{lacZ} promoter activity is observed in all plasmid model strains grown in any other carbon source, even glucose (Fig. \ref{lacZassay}).
Generally, in all of them we can see the same pattern - an increasing \tax{lacZ} promoter activity based on the carbon source used: glucose $<$ arabinose $<$ ribose $<$ galactose $<$ lactose.
The difference in expression between arabinose and ribose environment is relatively small but consistent among all strains and replicates.
Moreover, the mean expression levels in all these carbon sources depend not only on the promoter variant the strain has but also on the genetic background of the strain (see pZ\textunderscore D9-D9 vs. pZ\textunderscore D9-K12 and pZ2\textunderscore K12-D9 vs. pZ2\textunderscore K12-K12 in Fig. \ref{lacZassay}).
Note that some density curves (mainly those from glucose growth) do not have a log-normal shape which is common for all samples from lactose growth.

Presence of the fluorescence in some of these strains even during growth on non-lactose carbon source was confirmed by time-lapse fluorescent microscopy.
These results from growth on glucose as a sole carbon source are available in \hyperlink{micro}{Appendix 6}.
The results obtained this way were assessed only visually for now.
Strain pZ1\textunderscore K12-K12 seems to have the lowest GFP expression levels as well as the smallest fraction of fluorescent cells from the analysed plasmid-based models.
The other two strains pZ2\textunderscore K12-K12 and pZ2\textunderscore K12-D9 also seem to exhibit the same pattern which is seen from flow cytometry results.
It is that the latter strain has higher GFP expression levels than the former one and higher fraction of fluorescent cells as well.
As a negative control for GFP expression pUA66-K12 was included in the experiments as well (\hyperlink{micro}{Appendix 6}).
No fluorescence (above autofluorescence) is detectable using this approach.

We believe the disproportion in GFP expression between plasmid-based models and ASC662 strain under non-lactose environments is caused by multiple copies of \tax{lacZ} promoter present in the plasmid models.
Even though the plasmids used produce a low copy number \cite{zaslaver2006comprehensive}, it probably exceeds the number of LacI repressors present in the cells and thus enables the induction of some of the \tax{lacZ} promoters present.

\subsection{Activity of \tax{recA} promoters}
Mean GFP expression levels driven by \tax{recA} promoter are proportional to the concentration of Ciprofloxacin the cells are exposed to (Fig. \ref{recAassay}).
\begin{figure}[ht!]
  \centering
  \includegraphics[scale=0.25]{text/Pictures/recAassay.png}
    \caption{\tax{recA} promoter activity in increasing concentration of Ciprofloxacin. Strains pUA66-K12 and pU139-D9 are negative controls of GFP expression for plasmid-based models - a promoter-less plasmid present. The rest are strains with the same \tax{recA} promoter cloned upstream of the GFP gene in a plasmid - see Table \ref{strains}.}
    \label{recAassay}
\end{figure}
The induction under higher Ciprofloxacin concentrations is slightly less pronounced in pA\textunderscore K12-D9 than in pA\textunderscore K12-K12 (from 4.274 au, stdev: 0.219 to 4.779 au, stdev: 0.165, i.e., 3.20-fold change and from 4.315 au, stdev: 0.235 to 4.974 au, stdev: 0.178, i.e., 4.56-fold change from 2.0 to 4.0 ng/ml of Cip, respectively).
The mean fluorescence level is also over 1.5 times higher under 4.0 ng/ml of Ciprofloxacin in pA\textunderscore K12-K12 compared to pA\textunderscore K12-D9.

Another change based on the Ciprofloxacin concentration is observed in scatter plots.
At 2.0 ng/ml of Ciprofloxacin both the forward and side scatter values of the cells begins to increase (data not shown).
This is expected to happen due to cell filamentation under higher Ciprofloxacin exposure.
It also explains the shift in fluorescence in the negative controls of GFP expression, i.e., pUA66-K12 and pU139-D9, under the highest Ciprofloxacin concentration used (Fig. \ref{recAassay}).
Interestingly, the effect of filamentation is higher in SC1\textunderscore D9 background than in MG1655 (from 2.307 au, stdev: 0.197 to 2.589 au, stdev: 0.155, i.e., 1.91-fold change and from 2.332 au, stdev: 0.193 to 2.517 au, stdev: 0.169, i.e., 1.53-fold change from 2.0 to 4.0 ng/ml of Cip, respectively), which is the opposite of what we see in \tax{recA} promoter induction.
The higher filamentation effect of SC1\textunderscore D9 is likely due to its generally smaller cells compared to MG1655.


% 17.9.2018
% where is this name from?
\section{Construction of pADOUCH plasmid}
As shown above, \tax{lacZ} promoters cloned into plasmids were active even in non-lactose environments, but that is not the case if the GFP gene is integrated into a chromosome (Fig. \ref{lacZassay}).
We hypothesise that these differential results might be caused by the naturally low amounts of LacI repressor present in a cell.
When a cell makes multiple copies of \tax{lacZ} promoter, because of its presence on a plasmid, all LacI proteins become saturated, and some of the promoters escape the repression, even though the plasmids used should produce only several copies within a cell \cite{zaslaver2006comprehensive}.
To confirm this, we decided to place the \tax{placZ::GFP} sequences into chromosome, creating only one additional \tax{lacZ} promoter per a cell.

We have available pLW001 plasmid which is suitable for chromosomal integration (\hyperlink{pLW001}{Appendix 7}) however, this plasmid lacks a ribosome binding site (RBS).
Thus we constructed a new plasmid by replacing the GFP gene present in the pLW001 vector with the GFP sequence we have in our negative control plasmid pUA66 (\hyperlink{pUA66seq}{Appendix 1}) together with its strong RBS, some additional restriction sites and a transcriptional terminator (Fig. \ref{cloning}).
\begin{figure}[ht!]
  \centering
  \includegraphics[scale=0.41]{text/Pictures/Cloning.png}
    \caption{Scheme of pADOUCH plasmid construction by replacing GFP sequence from pLW001 vector by sequence from pUA66 plasmid.}
    \label{cloning}
\end{figure}

We successfully amplified a sequence of expected length (1.1 kb) in all 8 transformants tested (Fig. \ref{colonyPCR}), showing that successful integration had occurred.
\begin{figure}[ht!]
  \centering
  \includegraphics[scale=0.16]{text/Pictures/ColonyPCR.jpg}
    \caption{Electrophoresis of colony PCR using pUA66\textunderscore clon primers; M - marker; first 8 samples (1:5(1) - 1:3(2)) - tested clones; pLW001-Top10 and Top10 - negative controls with template DNA; pUA66-K12 - positive control; NEG - negative control without template DNA.}
    \label{colonyPCR}
\end{figure}
Sanger sequencing confirmed the presence of the desired insert in digested pLW001 vector.
Nevertheless, we identified four SNPs compared to the expected sequence in the GFP gene (Fig. \ref{1:3(1)seq}).
However, as both sequenced clones have exactly the same SNPs (\hyperlink{pADOUCHseq}{Appendix 8}) and all of these SNPs are synonymous (Fig. \ref{1:3(1)seq}), we inferred that the reference sequence of GFP gene in the pUA66 plasmid was inaccurate.
Complete revised sequence of pADOUCH vector can be found in \hyperlink{pADOUCHwhole}{Appendix 9}.
\begin{figure}[ht!]
  \centering
  \includegraphics[scale=0.26]{text/Pictures/pADOUCHseq.png}
    \caption{Excerpt of 1:3(1) clone sequence alignment with an expected pADOUCH reference. DNA sequence is dotted except for SNPs; protein sequence is always beneath its DNA sequence.}
    \label{1:3(1)seq}
\end{figure}

\cleardoublepage%%% keeps correct headings

\shorthandon{-} 
%%%%%%%%%%%%%%%%%%%%%%%%%%%%%%%%%%%%




\HeadingConclusion
\chapter*{Summary and future work}
\addcontentsline{toc}{chapter}{Summary}
\shorthandoff{-}


\cleardoublepage%%% keeps correct headings

\shorthandon{-}




\HeadingAppendix
\chapter*{List of Appendices}

\begin{enumerate}
\item Obrázek \ref{rrs_ML}
\item Obrázek \ref{rpoB_ML}
\item Obrázek \ref{rpoD_ML}
\item \hyperlink{SPARC.1}{Titulní strana sborníku SPARC 2016}
\item \hyperlink{SPARC.55}{Abstrakt z příspěvku v rámci SPARC 2016}
%\item \hyperlink{fenotyp.1}{Shluková analýza fenotypu}
%\item \hyperlink{ordination.1}{Ordinační analýza fenotypu}
%\item \hyperlink{genotyp.1}{Shluková analýza genotypu}
\end{enumerate}

\addcontentsline{toc}{chapter}{Appendix}
%\pageref{mypage}
%\hyperlink{fenotyp.1}{Dendrogram biochemie}
%\includepdf[link, linkname=fenotyp, pages=1, landscape, angle=270, fitpaper]{text_prace/Pictures/anal_vse_ward_jacc_colXYZA.pdf}

%\includepdf[link, linkname=ordination, pages=1, landscape, fitpaper]{text_prace/Pictures/anal_vse_PoCA_jaccard_big1.pdf}

%\includepdf[link, linkname=genotyp, pages=1, landscape, angle=270, fitpaper]{text_prace/Pictures/RPearson-reduction_XYZA_colFULL1.pdf}
\begin{figure}[h!!!]
  \centering
  \includegraphics[scale=0.50]{text/Pictures/160508_16S_ML_clustalW_Bootstrap-consensus.png}
	\caption{Fylogenetický strom na základě analýzy genu \tax{rrs} dle metody Maximum-likelihood}
	\label{rrs_ML}
\end{figure}
\pagebreak

\begin{figure}[h!!!]
  \centering
  \includegraphics[scale=0.50]{text/Pictures/160508_rpoB_multifasta_doplnek_ML_clustalW_Bootstrap-consensus.png}
	\caption{Fylogenetický strom na základě analýzy genu \tax{rpoB} dle metody Maximum-likelihood}
	\label{rpoB_ML}
\end{figure}
\pagebreak

\begin{figure}[h!!!]
  \centering
  \includegraphics[scale=0.50]{text/Pictures/160508_rpoD_multifasta_doplnek_ML_clustalW_Bootstrap-consensus.png}
	\caption{Fylogenetický strom na základě analýzy genu \tax{rpoD} dle metody Maximum-likelihood}
	\label{rpoD_ML}
\end{figure}
\clearpage

\includepdf[link, linkname=SPARC, pages=1, landscape, fitpaper]{text/Pictures/Sbornik-SPARC2016.pdf}

\includepdf[link, linkname=SPARC, pages=55, landscape, fitpaper]{text/Pictures/Sbornik-SPARC2016.pdf}

%Here is a \hyperlink{anal_vse_ward_jacc_color3.pdf.19}{hyperlink to page 19} of anal_vse_ward_jacc_color3.pdf.

%\medskip

%%%%%%%%%%%%%%%%%%%%%%%%%%%%%%%%%%%%
%%%%%%%%% GENERUJ TEXT %%%%%%%%%%%%%
%%%%%%%%%%%%%%%%%%%%%%%%%%%%%%%%%%%%

\cleardoublepage



\renewcommand{\bibname}{References}
\HeadingLiterature
\addcontentsline{toc}{chapter}{\bibname}

\bibliographystyle{apalike-mv}
\bibliography{Literature}

%%%%%%%%%%%%%%%%%%%%%%%%%%%%%%%%%%%%%%%%%%%%%%%%
%%%%%%%%%%% EMPTY PAGE TO THE END %%%%%%%%%%%%
%%%%%%%%%%%%%%%%%%%%%%%%%%%%%%%%%%%%%%%%%%%%%%%%

\newpage
\thispagestyle{empty}
\fancyhf{}
\newpage
\mbox{}

\end{document}
